
\thispagestyle{empty}

\vspace*{0.1in}

\begin{centering}
{\LARGE \PDAC\ Version \PDACVERSION}\\
{\Large \PDACNOTES}\\
\bigskip
{\large Authors: \UGAUTHORS} \\
\medskip
{\large Istituto Nazionale di Geofisica e Vulcanologia } \\
\bigskip
{\large \copyright 2003 INGV.
All Rights Reserved} \\
\bigskip
\end{centering}

  \rule{6in}{0.04in}				\\	\vspace{0.25in}

\section*{PDAC (\PDACNAME )\\
Non-Exclusive, Non-Commercial Use License}

\subsubsection*{Introduction}

INGV, in cooperation wtih CNR and CINECA,
has developed the \PDACNAME\ \PDAC\
to the purpose of scientific research.
\PDAC\ is then available free of charge for
non-commercial use by individuals, academic or research institutions,
upon completion and submission of the online registration form available 
from the \PDAC\ web site \PDACURL.

Commercial use of the \PDAC\ software, or derivative works based thereon,
REQUIRES A COMMERCIAL LICENSE. Commercial use includes: 
(1) integration of all or part of the Software into a product for sale, 
lease or license by or on behalf of Licensee to third parties, or 
(2) distribution of the Software to third parties that need it to 
commercialize product sold or licensed by or on behalf of Licensee.  
INGV will negotiate commercial-use licenses for PDAC upon request. 
These requests can be directed to \PDACADDRESS

\subsubsection*{Registration}

Individuals may register in their own name or with their institutional or
corporate affiliations. Registration information must include name, title, 
and e-mail of a person with signature authority to authorize and commit the
individuals, academic or research institution, or corporation as necessary 
to the terms and conditions of the license agreement.

All parts of the information must be understood and agreed to as part of
completing the form. Completion of the form is required before software 
access is granted. Pay particular attention to the authorized requester 
requirements above, and be sure that the form submission is authorized 
by the duly responsible person.

Registration will be administered by the \PDAC\ development team.

\newpage
\subsubsection*{\PDAC: \PDACNAME\ LICENSE AGREEMENT}

Upon execution of this Agreement by the party identified below (``Licensee''),
The Istituto Nazionale di Geofisica e Vulcanologia (``INGV''), on 
behalf of the PDAC Development Team (``PDAC Team''),
will provide \PDAC\ in Executable 
Code and/or Source Code form (``Software'') to Licensee, subject to 
the following terms and conditions. For purposes of this Agreement, 
Executable Code is the compiled code, which is ready to run on Licensee's 
computer. Source code consists of a set of files which contain the 
actual program commands that are compiled to form the Executable Code.

1. The Software is intellectual property owned by INGV, and all 
right, title and interest, including copyright, remain with INGV. 
INGV grants, and Licensee hereby accepts, a restricted, non-exclusive, 
non-transferable license to use the Software for academic, research 
and internal business purposes only e.g. not for commercial use 
(see Paragraph 7 below), without a fee. Licensee agrees to reproduce 
the copyright notice and other proprietary markings on all copies of 
the Software. Licensee has no right to transfer or sublicense the 
Software to any unauthorized person or entity. However, Licensee does 
have the right to make complementary works that interoperate with PDAC, 
to freely distribute such complementary works, and to direct others 
to the PDAC Team's server to obtain copies of PDAC itself.

2. Licensee may, at its own expense, modify the Software to make derivative
works, for its own academic, research, and internal business purposes.
Licensee's distribution of any derivative work is also subject to the same
restrictions on distribution and use limitations that are specified herein
for INGV Software. Prior to any such distribution the Licensee shall 
require the recipient of the Licensee's derivative work to first execute a 
license for PDAC with INGV in accordance with the terms and conditions 
of this Agreement. Any derivative work should be clearly marked and 
renamed to notify users that it is a modified version and not the original 
PDAC code distributed by INGV.

3. Except as expressly set forth in this Agreement, THIS SOFTWARE IS 
PROVIDED ``AS IS'' AND INGV MAKES NO REPRESENTATIONS AND EXTENDS NO 
WARRANTIES OF ANY KIND, EITHER EXPRESS OR IMPLIED, INCLUDING BUT NOT 
LIMITED TO WARRANTIES OR MERCHANTABILITY OR FITNESS FOR A PARTICULAR 
PURPOSE, OR THAT THE USE OF THE SOFTWARE WILL NOT INFRINGE ANY PATENT, 
TRADEMARK, OR OTHER RIGHTS. LICENSEE ASSUMES THE ENTIRE RISK AS TO 
THE RESULTS AND PERFORMANCE OF THE SOFTWARE AND/OR ASSOCIATED MATERIALS. 
LICENSEE AGREES THAT INGV SHALL NOT BE HELD LIABLE FOR ANY DIRECT, 
INDIRECT, CONSEQUENTIAL, OR INCIDENTAL DAMAGES WITH RESPECT TO ANY 
CLAIM BY LICENSEE OR ANY THIRD PARTY ON ACCOUNT OF OR ARISING FROM 
THIS AGREEMENT OR USE OF THE SOFTWARE AND/OR ASSOCIATED MATERIALS.

4. Licensee understands the Software is proprietary to INGV. 
Licensee agrees to take all reasonable steps to insure that the Software 
is protected and secured from unauthorized disclosure, use, or release 
and will treat it with at least the same level of care as Licensee would 
use to protect and secure its own proprietary computer programs and/or 
information, but using no less than a reasonable standard of care. 
Licensee agrees to provide the Software only to any other person or 
entity who has registered with INGV. If licensee is not registering 
as an individual but as an institution or corporation each member of 
the institution or corporation who has access to or uses Software must 
understand and agree to the terms of this license. If Licensee becomes 
aware of any unauthorized licensing, copying or use of the Software, 
Licensee shall promptly notify INGV in writing. Licensee expressly 
agrees to use the Software only in the manner and for the specific uses 
authorized in this Agreement.

5. By using or copying this Software, Licensee agrees to abide by the
copyright law and all other applicable laws of Italy including, but 
not limited to, export control laws and the terms of this license. 
INGV shall have the right to terminate this license immediately by 
written notice upon Licensee's breach of, or non-compliance with, any 
of its terms. Licensee may be held legally responsible for any copyright 
infringement that is caused or encouraged by its failure to abide by 
the terms of this license. Upon termination, Licensee agrees to destroy 
all copies of the Software in its possession and to verify such 
destruction in writing.

6. The user agrees that any reports or published results obtained 
with the Software will acknowledge its use by the appropriate citation 
as follows: 

\begin{quote}
 PDAC was developed by the Volcanic Simulation Group,
 at the Istituto Nazionale di Geofisica e Vulcanologia - Italy.
\end{quote}

Any published work which utilizes PDAC shall include the following references: 

\begin{quote}
  Neri et al., ``Multiparticle simulation of collapsing volcanic columns 
  and pyroclastic flow'', {\em J.Geophys.Res}, vol.108, NO.B4 (2003)
\end{quote}

\begin{quote}
  Neri et al., ``\PDAC\ Reference Manual'' (2004)
\end{quote}

Electronic documents will include a direct link to the official PDAC page:

\begin{quote}
\PDACURL
\end{quote}

One copy of each publication or report will be supplied to INGV 
through Dr. Augusto Neri, at the addresses listed below in Contact 
Information.

7. Should Licensee wish to make commercial use of the Software, Licensee 
will contact INGV (\PDACADDRESS) to negotiate an appropriate 
license for such use. Commercial use includes: (1) integration of all 
or part of the Software into a product for sale, lease or license by or 
on behalf of Licensee to third parties, or (2) distribution of the 
Software to third parties that need it to commercialize product sold or 
licensed by or on behalf of Licensee.

8. Because substantial funds in the development of PDAC have been provided
by the European Commission, Gruppo Nazionale per la Vulcanologia (Italy)
and Ministero dell'Istruzione, dell'Universit\`a e della Ricerca (Italy),
any use, distribution or sublicense of the Software is subject to the specific
policies of these institutions.

9. PDAC is being distributed as a research and teaching tool and as such, 
the PDAC Team encourages contributions from users of the code that might, at 
INGV' sole discretion, be used or incorporated to make the basic 
operating framework of the Software a more stable, flexible, and
useful product. Licensees that wish to contribute their code to become 
an internal portion of the Software may be required to sign an 
``Agreement Regarding Contributory Code for PDAC Software''
before INGV can accept it (contact \PDACADDRESS\  for a copy).

\newpage
\subsubsection*{Contact Information}

The best contact path for licensing issues is by e-mail to \PDACADDRESS\ 
or send correspondence to:
\begin{verse}
                             PDAC Development Team\\
                             Istituto Nazionale di Geofisica e Vulcanologia\\
			     Centro di modellistica fisica e pericolosit\`a 
			     dei processi vulcanici\\
			     Sezione Sismologia e tettonofisica\\
                             32, Via della Faggiola\\
			     56126, Pisa - Italy\\
                             FAX: +39-0508311942
\end{verse}


