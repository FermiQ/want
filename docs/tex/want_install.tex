%%%%%%%%%%%%%%%%%%%%%%%%%%%%%%%%%%%%%%%%%%%%%%%%%%%%%%%%%%
%  Copyright (C) 2005 WanT group                         %
%  This file is distributed under the terms of the       %
%  GNU General Public License.                           %
%  See the file `License'  in the root directory of      %
%  the present distribution,                             %
%  or http://www.gnu.org/copyleft/gpl.txt                %
%%%%%%%%%%%%%%%%%%%%%%%%%%%%%%%%%%%%%%%%%%%%%%%%%%%%%%%%%%

\thispagestyle{empty}
\section{Installation procedure}\label{section:install}

\noindent \underline {NOTES}: (i) The present version of the code
adopts the installation procedure of the \QUANTUMESPRESSO\ distribution (for more details see
also \QUANTUMESPRESSOURL). 
Details are also reported in the {\tt \~{}want/docs/README.install} file. \\

\noindent Installation is a two-step procedure:
%
%
\begin{enumerate}
\item \texttt{cd} to the top directory of the \WANT{} tree,
and issue this command at the shell
prompt:\\
{\tt ./configure [$\langle$options$\rangle$] } \\
use the option {\tt -h} or {\tt --help} to get a full description of the options.
\item Then run:\\
     {\tt make $\langle$target$\rangle$ }
\end{enumerate}
%
%
\noindent
where {\tt $\langle$target$\rangle$} is one (or more) of the following (typically {\tt all}):\\[10pt]
{\tt wannier, transport, utility, \\
all, deps, clean, wash}. \\[15pt] 

\noindent
Running {\tt make} without arguments prints a short manual.
Cross-compilation is not currently supported.

%----------------------------------------------------------------------
%First step : configuring
%----------------------------------------------------------------------

\subsection{Step one: configuring} {\tt configure} is a GNU-style configuration script,
automatically generated by GNU Autoconf. If you want to play
with it, its source file is {\tt \~{}want/config/configure.ac}; you may also
want to edit {\tt \~{}want/config/make.sys.in} or {\tt \~{}want/config/configure.h.in}. 
It generates the following files: \\

%
%
\begin{tabular}{ll}
  \texttt{\~{}want/make.sys}            &     {compilation settings and flags}\\
  \texttt{\~{}want/include/*.h}         &     {header files to be included during compilation}\\
\end{tabular}
%
%
\\

\noindent To force using a particular compiler, or compilation
flags, or libraries, you may set the appropriate environment
variables when running the configuration script.  For example:

%
%
\begin{description}
  \item {\tt ./configure CC=gcc CFLAGS=-O3 }
\end{description}
%
%

\noindent Some of those environment variables are: \\

%
%
\begin{tabular}{ll}
  \texttt{TOPDIR}       &{: top directory of the \WANT{} tree (defaults to `pwd`)}\\
  \texttt{F90, F77, CC} &{: Fortran 90, Fortran 77, and C compilers}\\
  \texttt{CPP}          &{: source file preprocessor (defaults to "\$CC -E")}\\
  \texttt{LD}           &{: linker (defaults to \$F90)}\\
  \texttt{CFLAGS, FFLAGS,}  &  \\
  \texttt{F90FLAGS, CPPFLAGS, LDFLAGS} &{: compilation flags}\\
  \texttt{LIBDIRS}      &{: extra directories to search for libraries (see below)}
\end{tabular}
%
%
\\[15pt]

\noindent
To compile serially: \\
\begin{tabular}{l}
    \qquad {\tt ./configure --disable-parallel}
\end{tabular}
\\[15pt]

\noindent
To compile using internal blas and lapack libraries : \\
\begin{tabular}{l}
    \qquad {\tt ./configure --with-internal-libs} \qquad or \\
    \qquad {\tt ./configure --with-internal-blas --with-internal-lapack}
\end{tabular}
\\[15pt]

\noindent
To switch on the ETSF-IO support (used to interface WanT to Abinit): \\
\begin{tabular}{l}
\qquad {\tt ./configure --enable-etsf-io} \qquad or \\
\qquad {\tt ./configure --with-etsf-io}
\end{tabular}
\\[15pt]


\noindent You should always be able to compile the \WANT{} suite of
programs without having to edit any of the generated files.  If
you ever have to, that should be considered a bug in the
configuration script and you are encouraged to submit a bug
report.\\

\noindent \underline {IMPORTANT}: \WANT{} can take advantage of
several
optimized numerical libraries:\\
\begin{itemize}
 \item[-] ESSL on AIX systems (shipped by IBM).
 \item[-] MKL together with Intel compilers (shipped by Intel,
           free for non-commercial use)
 \item[-] ATLAS (freely downloadable from
           \texttt{ http://math-atlas.sourceforge.net })
 \item[-] FFTW (freely downloadable from
           \texttt{ http://www.fftw.org})\\
\end{itemize}

\noindent The configuration script attempts to find those
libraries, but may fail if they have been installed in
non-standard locations. You should look at the LIBS environment
variable (either in the output of the configuration script, or in
the generated {\tt make.sys}) to check whether all available
libraries were found.
If any libraries weren't found, you can rerun the
configuration script and pass it a list of directories to search,
by setting the environment variable LIBDIRS; directories in the
list must be
separated by spaces.  For example:
%
%
\begin{description}
  \item \texttt{ ./configure LIBDIRS="/opt/intel/mkl/mkl61/lib/32
  /usr/local/lib/fftw-2.1.5" }
\end{description}
%
%

\noindent If this still fails, you may set the environment
variable LIBS manually and retry.  For example:
%
%
\begin{description}
  \item \texttt{ ./configure LIBS="-L/cineca/prod/intel/lib -lfftw -llapack
  -lblas" }
\end{description}
%
%

\noindent Beware that in this case, you must specify \textbf{all} the
libraries that you want to link to.  The configuration script will
blindly accept the specified value, and will \textbf{not} search for any
extra library.\\


\newpage
%----------------------------------------------------------------------
%Second step : compiling
%----------------------------------------------------------------------
\subsection{Step two: compiling}
\noindent Here is a list of available compilation targets: \\

%
%
\begin{tabular}{llll}
  \texttt{make wannier}    &  compile & \texttt{disentangle.x} &(step 1)\\
  \texttt{}                &          & \texttt{wannier.x}     &(step 2)\\
  \texttt{}                &          & \texttt{dos.x}         &(post proc)\\
  \texttt{}                &          & \texttt{bands.x}       &(post proc)\\
  \texttt{}                &          & \texttt{plot.x}        &(post proc)\\
  \texttt{}                &          & \texttt{blc2wan.x}     &(post proc)\\
  \texttt{}                &          & \texttt{cmplx\_bands.x} &(post proc)\\
  \texttt{make transport}  &  compile & \texttt{conductor.x}   &(step 3)\\
  \texttt{}                &          & \texttt{current.x}     &(post proc)\\
  \texttt{make embed}      &          & \texttt{embed.x}       &(step 3, experimental)\\
  \texttt{make utility}    &  compile & auxiliary tools        &\\
  \texttt{make all}        &          & \texttt{make wannier + transport +} & \\
                           &          & \texttt{utility + embed} & \\
\end{tabular} 
\\[15pt]
\begin{tabular}{lll}
  \texttt{make libwant}    &  compile & \WANT{} basic libs       \\
  \texttt{make libctools}  &  compile & c-based support           \\
  \texttt{make libextlibs} &  compile & the internal version of some std external libraries         \\
                           &          & (blas, lapack, etc). It is automatically issues when those  \\
                           &          & libs are not found \\
  \texttt{make libplugins} &  compile & plug-ins libraries (such as NetCDF and ETSF-IO)  \\
  \texttt{make clean}      &  remove  & Object files, libs and executables     \\ 
  \texttt{make clean\_test}&  remove  & test output files     \\ 
  \texttt{make wash}       &  remove  & configuration files too   \\
\end{tabular}
%
%
\\[15pt]

\noindent \underline {IMPORTANT}: If you change any compilation or
precompilation option after a previous (successful or failed)
compilation, you must run \texttt{make clean} before recompiling,
unless you know exactly which routines are affected by the changed
options and how to force their recompilation.

%%%%%%%%%%%%%%%%%%%%%%%%%%%%%%%%%%%%%%%%%%%%%%%%%%%%%%%%%%%%%%%

\newpage
\subsection{List of directories}
Within the top directory of the \WANT{} tree there are the
following directories: \\

\begin{tabular}{lll}
{\bf docs}          & $\qquad$  &   documentation files and manual \\
{\bf config}        & $\qquad$  &   configuration tools \\
{\bf bin}           & $\qquad$  &   links to all the executables \\
{\bf include}       & $\qquad$  &   header files *.h \\
{\bf wannier}       & $\qquad$  &   source files for the Wannier functions suite \\
{\bf transport}     & $\qquad$  &   source files for the transport code \\
{\bf embed}         & $\qquad$  &   source files for the embed code \\
{\bf utility}       & $\qquad$  &   source files for auxiliary utilities\\
{\bf libs}          & $\qquad$  &   source files for basic common libraries \\
{\bf extlibs}       & $\qquad$  &   source files external libraries \\
{\bf plugins}       & $\qquad$  &   source files plugins libraries \\
{\bf ctools}        & $\qquad$  &   source files for c-support\\
{\bf scripts}       & $\qquad$  &   basic scripts\\
{\bf tests}         & $\qquad$  &   tutorial examples for the use of \WANT{}\\
\end{tabular}

