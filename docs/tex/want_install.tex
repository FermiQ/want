%%%%%%%%%%%%%%%%%%%%%%%%%%%%%%%%%%%%%%%%%%%%%%%%%%%%%%%%%%
%  Copyright (C) 2005 WanT group                         %
%  This file is distributed under the terms of the       %
%  GNU General Public License.                           %
%  See the file `License'  in the root directory of      %
%  the present distribution,                             %
%  or http://www.gnu.org/copyleft/gpl.txt                %
%%%%%%%%%%%%%%%%%%%%%%%%%%%%%%%%%%%%%%%%%%%%%%%%%%%%%%%%%%

\thispagestyle{empty}
\section{Installation procedure}\label{section:install}

\noindent \underline {NOTES}: (i) The present version of the code
adopts the installation procedure of the \PWSCF package (for more details see
also \PWSCFURL). (ii) This installation procedure is still
experimental, and only a limited number of architectures are
currently supported.
Details are also reported in the {\tt \$TOPDIR/docs/README.install} file,
where {\tt \$TOPDIR} is the top directory of the \WANT\ source tree. \\

\noindent Installation is a two-step procedure:
%
%
\begin{enumerate}
\item \texttt{cd} to the top directory of the \WANT\ tree,
and issue this command at the shell
prompt:\\
{\tt ./configure [$\langle$options$\rangle$] } \\
use the option {\tt -h} or {\tt --help} to get a brief manual.
\item Now run:\\
     {\tt make $\langle$target$\rangle$ }
\end{enumerate}
%
%
where {\tt $\langle$target$\rangle$} is one (or more) of the following:
{\tt wannier, transport, libwant, libiotk, all, clean, wash}. Running
{\tt make} without arguments prints a short manual.
Cross-compilation is not currently supported.

%----------------------------------------------------------------------
%First step : configuring
%----------------------------------------------------------------------

\subsection{Step one: configuring} {\tt configure} is a GNU-style configuration script,
automatically generated by GNU Autoconf. If you want to play
with it, its source file is {\tt \$TOPDIR/conf/configure.ac}; you may also
want to edit {\tt \$TOPDIR/conf/make.sys.in} or {\tt \$TOPDIR/conf/configure.h.in}. 
It generates the following files: \\

%
%
\begin{tabular}{ll}
  \texttt{\$TOPDIR/make.sys}            &     {compilation settings and flags}\\
  \texttt{\$TOPDIR/include/configure.h} &     {export configuration flags to a inc file}\\
  \texttt{\$TOPDIR/conf/configure.msg}  &     {summary of configuration}\\
  \texttt{\$TOPDIR/*/make.depend}       &     {dependencies, in each source dir} \\
\end{tabular}
%
%
\\

\noindent Files {\tt make.depend} are actually generated by the
{\tt \$TOPDIR/conf/makedeps.sh} shell script. If you modify the program sources,
you might have to rerun it or, alternatively, it can be invoked with
{\tt ./configure -d}.  \\

\noindent To force using a particular compiler, or compilation
flags, or libraries, you may set the appropriate environment
variables when running the configuration script.  For example:

%
%
\begin{description}
  \item {\tt ./configure CC=gcc CFLAGS=-O3 LIBS="-llapack -lblas
  -lfftw" }
\end{description}
%
%

\noindent Some of those environment variables are: \\

%
%
\begin{tabular}{ll}
  \texttt{TOPDIR}       &{: top directory of the \WANT\ tree (defaults to `pwd`)}\\
  \texttt{F90, F77, CC} &{: Fortran 90, Fortran 77, and C compilers}\\
  \texttt{CPP}          &{: source file preprocessor (defaults to "\$CC -E")}\\
  \texttt{LD}           &{: linker (defaults to \$F90)}\\
  \texttt{CFLAGS, FFLAGS,}  &  \\
  \texttt{F90FLAGS, CPPFLAGS, LDFLAGS} &{: compilation flags}\\
  \texttt{LIBDIRS}      &{: extra directories to search for libraries (see below)}\\
\end{tabular}
%
%
\\

\noindent You should always be able to compile the \WANT\ suite of
programs without having to edit any of the generated files.  If
you ever have to, that should be considered a bug in the
configuration script and you are encouraged to submit a bug
report.\\

\noindent \underline {IMPORTANT}: \WANT\ can take advantage of
several
optimized numerical libraries:\\
\noindent -- ESSL on AIX systems (shipped by IBM)\\
\noindent -- MKL together with Intel compilers (shipped by Intel,
free for non-commercial use)\\
\noindent -- ATLAS (freely downloadable from
   \texttt{ http://math-atlas.sourceforge.net })\\
\noindent -- FFTW (freely downloadable from
   \texttt{ http://www.fftw.org})\\

\noindent The configuration script attempts to find those
libraries, but may fail if they have been installed in
non-standard locations. You should look at the LIBS environment
variable (either in the output of the configuration script, or in
the generated {\tt make.sys}) to check whether all available
libraries were found.
If any libraries weren't found, you can rerun the
configuration script and pass it a list of directories to search,
by setting the environment variable LIBDIRS; directories in the
list must be
separated by spaces.  For example:
%
%
\begin{description}
  \item \texttt{ ./configure LIBDIRS="/opt/intel/mkl/mkl61/lib/32
  /usr/local/lib/fftw-2.1.5" }
\end{description}
%
%

\noindent If this still fails, you may set the environment
variable LIBS manually and retry.  For example:
%
%
\begin{description}
  \item \texttt{ ./configure LIBS="-L/cineca/prod/intel/lib -lfftw -llapack
  -lblas" }
\end{description}
%
%

\noindent Beware that in this case, you must specify \textbf{all} the
libraries that you want to link to.  The configuration script will
blindly accept the specified value, and will \textbf{not} search for any
extra library.\\

\noindent If you want to use the FFTW library, the \texttt{fftw.h}
include file is also required.  If the configuration script wasn't
able to find it, you can specify the correct directory in the
INCLUDEFFTW environment variable. For example:
%
%
\begin{description}
  \item {\tt ./configure INCLUDEFFTW="/cineca/lib/fftw-2.1.3/fftw" }
\end{description}
%
%

%----------------------------------------------------------------------
%Second step : compiling
%----------------------------------------------------------------------
\subsection{Step two: compiling}
\noindent Here is a list of available compilation targets: \\

%
%
\begin{tabular}{llll}
  \texttt{make wannier}    &  compile & \texttt{disentangle.x} &(step 1)\\
  \texttt{}                &          & \texttt{wannier.x}     &(step 2)\\
  \texttt{}                &          & \texttt{bands.x}       &(post proc)\\
  \texttt{}                &          & \texttt{plot.x}        &(post proc)\\
  \texttt{}                &          & \texttt{blc2wan.x}     &(post proc)\\
  \texttt{make transport}  &  compile & \texttt{conductor.x}   &(step 3)\\
  \texttt{make all}        &          & \texttt{make wannier + transport} & \\
  \texttt{make libwant}    &  compile & \WANT{} basic libs       &        \\
  \texttt{make libiotk}    &  compile & Input-Output toolkit lib (iotk)  &        \\
  \texttt{make clean}      &  remove  & Object files, libs and executables &    \\ 
  \texttt{make clean\_test}&  remove  & test output files &    \\ 
  \texttt{make wash}       &  remove  & Configuration files too        &  
\end{tabular}
%
%
\\

\noindent \underline {IMPORTANT}: If you change any compilation or
precompilation option after a previous (successful or failed)
compilation, you must run \texttt{make clean} before recompiling,
unless you know exactly which routines are affected by the changed
options and how to force their recompilation.

%%%%%%%%%%%%%%%%%%%%%%%%%%%%%%%%%%%%%%%%%%%%%%%%%%%%%%%%%%%%%%%

\subsection{List of directories}
Within the top directory of the \WANT\ tree there are the
following directories: \\

\newdimen\descindent \descindent = 8pc
{\noindent \leftskip = \descindent \parskip = .5\baselineskip
\llap{\hbox to \descindent{\bf conf\hfil}}%
configuration files\par

\noindent\llap{\hbox to \descindent{\bf docs\hfil}}%
documentation files and manuals \par

\noindent\llap{\hbox to \descindent{\bf iotk\hfil}}%
source files for the iotk library (input-output toolkit, by G. Bussi) \par

\noindent\llap{\hbox to \descindent{\bf wannier\hfil}}%
source files for the Wannier functions suite 
/par

\noindent\llap{\hbox to \descindent{\bf transport\hfil}}%
source files and modules for transport code
\par

\noindent\llap{\hbox to \descindent{\bf bin\hfil}}%
links to all the executables \par

\noindent\llap{\hbox to \descindent{\bf include\hfil}}%
include files *.h \par

\noindent\llap{\hbox to \descindent{\bf libs\hfil}}%
source files for basic common libraries \par

\noindent\llap{\hbox to \descindent{\bf tests\hfil}}%
tutorial examples for the use of \WANT\ suite 
\par

\noindent\llap{\hbox to \descindent{\bf utility\hfil}}%
general utility and tools. \par}


