%%%%%%%%%%%%%%%%%%%%%%%%%%%%%%%%%%%%%%%%%%%%%%%%%%%%%%%%%%
%  Copyright (C) 2005 WanT group                         %
%  This file is distributed under the terms of the       %
%  GNU General Public License.                           %
%  See the file `License'  in the root directory of      %
%  the present distribution,                             %
%  or http://www.gnu.org/copyleft/gpl.txt                %
%%%%%%%%%%%%%%%%%%%%%%%%%%%%%%%%%%%%%%%%%%%%%%%%%%%%%%%%%%

\thispagestyle{empty}
\section{How to setup input files}\label{sec:input}

\noindent According to the methodological scheme of Sec.~\ref{sec:run},
it is necessary to use separate input files at
the different step of the \WANT{} procedure.\\

\noindent Input files are organized using several {\tt NAMELIST}s,
followed by other fields with more massive data {\tt CARDS}. Namelists are
begin with the flag {\tt \&NAMELIST } and end with the
"$/$'' bar. The order of variables within a namelist is
arbitrary. Most variables have default values mandatory.
If a variable is not explicitly defined in the input file,
its default value is assumed. Other variables are mandatory and must be
always supplied.
In the following we report the list and the description
of the details of each required input file.

\subsubsection{Input for DFT-PW calculations}
%
\noindent \WANT{} is currently interfaced to the codes in the \QUANTUMESPRESSO distribution. 
For the description of the input for steps 1-2 (Sec. \ref{sec:run})
and for further details see {\it e.g.} the \PWSCF{} manual at \PWSCFURL.\\
The input description to use {\tt pw\_export.x} can also be found 
in the file {\tt README\_pwexport.input} in the {\tt docs} directory of
the \WANT{} distribution.

%
\subsubsection{Input for codes in the \WANT{} suite}
%
\noindent
Textual {\tt README} files describing how to setup input files
for all the codes distributed within \WANT{} can be found in the 
{\tt docs} directory: \\

\begin{tabular}{@{\hspace{10pt}}l}
    {\tt README\_wannier.input}  \\
    {\tt README\_conductor.input} \\
    {\tt README\_bands.input} \\
    {\tt README\_dos.input} \\
    {\tt README\_plot.input} \\
    {\tt README\_blc2wan.input}  \\
\end{tabular}

