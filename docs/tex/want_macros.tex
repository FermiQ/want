%%%%%%%%%%%%%%%%%%%%%%%%%%%%%%%%%%%%%%%%%%%%%%%%%%%%%%%%%%
%  Copyright (C) 2005 WanT group                         %
%  This file is distributed under the terms of the       %
%  GNU General Public License.                           %
%  See the file `License'  in the root directory of      %
%  the present distribution,                             %
%  or http://www.gnu.org/copyleft/gpl.txt                %
%%%%%%%%%%%%%%%%%%%%%%%%%%%%%%%%%%%%%%%%%%%%%%%%%%%%%%%%%%
%
% generally useful macros
%
\newcommand{\htmlimage}[1] {}
\newcommand{\REFAND} {\&}
\newcommand{\ETALNP}{\mbox{\it et al}}
\newcommand{\ETAL}{\mbox{\ETALNP{\it.}}}
\newcommand{\eqnref}[1] {\mbox{eq (\ref{#1})}}
\newcommand{\mycite}[2] {\cite{#2}}

%
% macros for math conventions in the theory description.
%
\newcommand{\cc}{{*}}
\newcommand{\br}{{\vec{r}}}
\newcommand{\bpsi}{{\vec{\psi}}}
\newcommand{\bphi}{{\vec{\phi}}}
\newcommand{\brho}{{\bar \rho}}
\newcommand{\bk}{{\vec{k}}}
\newcommand{\bR}{{\vec{R}}}
\newcommand{\bF}{{\vec{F}}}
\newcommand{\dpsi}{{\delta\psi}}
\newcommand{\ttvu}{{\bpsi^{(k+1)}}}
\newcommand{\tepsilon}{{\epsilon^{(k)}}}
\newcommand{\epsilonkj}{{\epsilon^{(k)}_j}}
\newcommand{\psij}{{\psi_j}}
\newcommand{\psii}{{\psi_i}}
\newcommand{\epsilonj}{{\epsilon_{j}}}
\newcommand{\psik}{{{\bpsi}^{(k)}}}
\newcommand{\psikj}{{{\bpsi}^{(k)}_j}}
\newcommand{\psikone}{{{\bpsi}^{(k)}_1}}
\newcommand{\psikN}{{{\bpsi}^{(k)}_N}}
\newcommand{\dpsik}{{{{\delta\bpsi}^{(k)}}}}
\newcommand{\dpsikj}{{{{\delta\bpsi}^{(k)}_j}}}
\newcommand{\dpsikone}{{{{\delta\bpsi}^{(k)}_1}}}
\newcommand{\dpsikN}{{{{\delta\bpsi}^{(k)}_N}}}
\newcommand{\rkj}{{{\br}^{(k)}_j}}
\newcommand{\Psik}{{\Psi^{(k)}}}
\newcommand{\Phik}{{\Phi^{(k)}}}

\newcommand{\WANT} {{\sc WanT}{}}
\newcommand{\WANTSUBTITLE}{{\sc an integrated approach to ab initio electronic
transport from maximally-localized Wannier functions}}
\newcommand{\WANTNAME} {Wannier Transport}
\newcommand{\WANTDATE} {\today}
%\newcommand{\WANTDATE} {October 2005}
\newcommand{\UGAUTHORS} {
A. Ferretti, A. Calzolari, C.~Cavazzoni and M. Buongiorno
Nardelli}
\newcommand{\WANTVERSION} {2.0.0}
\newcommand{\WANTURL} {\texttt{http://www.wannier-transport.org}}
%
% other cited codes
\newcommand{\PWSCF} { \textsc{PWscf}{ }}
\newcommand{\PWSCFURL} {\texttt {http://www.pwscf.org}{}}
\newcommand{\ABINIT} { \textsc{Abinit}{}}


%
% macros for style conventions when describing the program.
%

% name of class or object in program
\newcommand{\OBJ}[1] {{\tt#1}}

% name of file
\newcommand{\FIL}[1] {{\it #1}}

% name of directory
\newcommand{\DIR}[1] {$<${\it #1}$>$}

% function arguments
\newcommand{\FA}[2] {{\rm{\bf#1}\ {\it#2}}}

% global function name
\newcommand{\FN}[3] {{\rm\bf#1}\ {\tt #2(}#3{\tt)}}

% class member function name
\newcommand{\FNO}[4] {{\rm\bf#2}\ \OBJ{#1::}{\tt#3(}#4{\tt)}}

% list item, for optional components, parameters, etc.
\newcommand{\TTLISTITEM}[1] {\item {\tt #1} \\}
\newcommand{\RMLISTITEM}[1] {\item {\rm #1} \\}
\newcommand{\BOLDLISTITEM}[1] {\item {\bf #1} \\}
\newcommand{\EMLISTITEM}[1] {\item {\em #1} \\}
\newcommand{\LISTITEM}[1] {\RMLISTITEM{#1}}

%
% other generally useful macros
%

\newcommand{\UGTITLE} {User's Guide}
\newcommand{\RMTITLE} {Reference Manual}
\newcommand{\UGDESC} {\noindent This {\em\UGTITLE\/} describes how to run and use the various
features of the integrated \WANT\ approach. This guide includes a
description of the capabilities of the program, how to use these
capabilities, the necessary input files and formats, and how to
run the program on both serial and parallel machines.}

\newcommand{\RMDESC} {
The \WANT\ (\WANTNAME)\ {\em\RMTITLE\/} is intended for the user
who wants to overview the model equations and the numerical
technique or for the programmer who wants to cooperate to the
\WANTC\ development. The manual describes how the model equations
are discretized and solved, provides the explicative list of all
variables used in the numerical code and gives a brief description
of the parallelization strategy, so that the advanced user can
optimize the code efficiency on several computational platforms.}

\newcommand{\PG}{\WANT\ {\it Programmer's Guide\/}}
\newcommand{\UG}{\WANT\ {\it User's Guide\/}}
\newcommand{\RM}{\WANT\ {\it Reference Manual\/}}
\newcommand{\prettypar}{

\smallskip

}

\newcommand{\eg}{{\it e.g.\/}}
\newcommand{\ie}{{\it i.e.\/}}

\newcommand{\KEY}[1]{{\tt #1}}
\newcommand{\IKEY}[1]{{\tt #1\index{#1 psfgen command}}}
\newcommand{\OKEY}[1]{$[${\tt #1}$]$}
\newcommand{\ARG}[1]{$<${\em #1}$>$}
\newcommand{\OARG}[1]{$[${\em #1}$]$}
\newcommand{\ARGDEF}[2]{$<${\em #1}$>$: #2}
\newcommand{\KEYDEF}[2]{{\tt #1}: #2}
\newcommand{\COMMAND}[4]{%
  #1 \\ {\bf Purpose:} #2 \\ {\bf Arguments:} #3 \\ {\bf Context:} #4 }

\newcommand{\icommand}[1]{#1\index{#1 command}}

\newcommand{\WANTCONF}[4]{%
%  \addcontentsline{toc}{subparagraph}{#1}%
  {\bf \tt #1 } $<$ #2 $>$ \\%
  \index{#1 parameter}
  {\bf Acceptable Values: } #3 \\%
  {\bf Description: } #4%
}

\newcommand{\WANTCONFWDEF}[5]{%
%  \addcontentsline{toc}{subparagraph}{#1}%
  {\bf \tt #1 } $<$ #2 $>$ \\%
  \index{#1 parameter}
  {\bf Acceptable Values: } #3 \\%
  {\bf Default Value: } #4 \\%
  {\bf Description: } #5%
}
