%%%%%%%%%%%%%%%%%%%%%%%%%%%%%%%%%%%%%%%%%%%%%%%%%%%%%%%%%%
%  Copyright (C) 2005 WanT group                         %
%  This file is distributed under the terms of the       %
%  GNU General Public License.                           %
%  See the file `License'  in the root directory of      %
%  the present distribution,                             %
%  or http://www.gnu.org/copyleft/gpl.txt                %
%%%%%%%%%%%%%%%%%%%%%%%%%%%%%%%%%%%%%%%%%%%%%%%%%%%%%%%%%%

\thispagestyle{empty}
\section{What to do when things go wrong?}
\label{sec:troubleshoot}
%
This section is not intended to be complete. Here we report
a summary of the most frequent and well-known problems in the
day-by-day practice with \WANT, and some tentative suggestions to
solve them. Please, report any better solution or explanation you
find to the maintainer of this manual to make it more detailed.

%%%%%%%%%%%%%%%%%%%%%%%%%%%%%%%%%%%%%%%%%%%
\subsection{When do things go really wrong ?}
First it is necessary to understand which behaviors should be considered buggy and
which may be conversely related to some failures of the implemented algorithms.
This section is devoted to guide the user to understand whether the code is
properly running or not. \\
%
%
\begin{itemize}
\item   {\bf Stops and crashes.} \\
        When the code stops, it is expected either to have reached the end
        of the calculation (which is marked in the output file by a summary
        of the timing) or to print
        out an error message and to give a fortran stop. Any different behavior
        should be considered a bug and should be reported
        (obviously it may be related to machine dependent problems,
        independently of \WANT).

\item   {\bf Problems connected to memory usage.} \\
        Since the code is still serial, no distribution of the memory among
        CPUs is performed and crashes may result. In these cases, both an
        error message during allocations or a more drastic error (even segmentation
        fault) may occur. The largest amount of memory is used by
        {\tt disentangle.x} (which manage wave-functions), while {\tt wannier.x}
        does not require that much. Post-processing ({\tt bands.x, blc2wan.x})
        do not have special requirement too, while {\tt plot.x} needs a sensible
        amount of memory, about half of {\tt disentangle.x}.

\item   {\bf Convergence in disentanglement minimization.} \\
        The iterative procedure in {\tt disentangle.x} is quite robust.
        Non-monotonic behaviors of the invariant spread are unexpected
        and are probably related to algorithm failures. Although, convergence
        may be very slow, especially when some parameters are not properly set
        (too many empty states in the main energy window, strange numbers of
        requested WFs...) .

\item   {\bf Convergence in Wannier localization.} \\
        In the case of {\tt wannier.x}, convergence is less straightforward.
        Particularly for large systems, it may happen that the total spread first 
        decreases for several iterations and then it suddenly jumps to a much higher
        value.  This cycle may be repeated several times.
        From the expressions for the spread funcitonal in Ref.~\cite{nicola}, we see that
        many evaluations of the imaginary part of complex logarithms are needed. 
        These are the phases of the arguments, which in the present case are the overlap
        integrals. Since logarithm in the complex plane
        is a multivalued function, sudden jumps in the total spread (towards larger
        values) can be related to changes in the logarithm branch. For instance, a phase
        which is increasing from $0$ to $2\pi$ is suddenly taken to 0 (
        {\it i.e.} to a different branch) when crossing the $2\pi$ value.

        Such behaviors should be considered ordinary: the user is suggested to increase
        the maximum number of iterations (50000 steps may be fully acceptable)
        when running {\tt wannier.x} . Since this problem can be demonstrated to be 
        connected with the use of
        large $\mathbf{k}$-point meshes, it should luckily leave unaffected large scale
        calculations (where we usually have large number of bands in a very small BZ).

\item   {\bf How can I understand when my Wannier functions are well behaved ?} \\
        Directly from the definition of {\it maximally localized} WFs, we basically
        are interested in {\it localized} orbitals spanning the original subspace
        of Bloch states. Our measure of localization (the spread functional $\Omega$)
        is anyway a global property of the WF set as a whole.
        This means that even if we are reaching lower and lower
        values of the spread, the set of WFs we found is a good one only if {\it each}
        WF is localized.

        If our application of WFs is based on their localization property, then
        a single WF not properly localized can make the whole set useless. This is
        the case of the current application to transport (but it is not
        for instance that of spontaneous polarization, which is a property of the 
        occupied manyfold).

        Taking in mind the above discussion,
        it is possible to identify the following criteria:
        %
        %
        \begin{itemize}
        \item    Typical values for the average invariant spreads are
                 lower than $7-10\,\, \text{Bohr}^2$. Note that this value is a sort of
                 lower bound for the final spreads.
        \item    the spread of each WF should be in a reasonable range
                 ( $< 15-20\,\, \text{Bohr}^2$). Note that the current version of
                 \WANT{} fully
                 adopts Bohr units, while older versions ({\it e.g.} v1.0) use $\AA^2$
                 for the spread in the {\tt wannier.x} output file.
        \item    Since well-localized WFs are expected (by conjecture) to be almost real,
                 their Hamiltonian matrix elements should be nearly real too.
                 See the end of {\tt wannier.x} output file where these matrix elements
                 are reported.
        \item    The spatial decay of the Hamiltonian $H(\mathbf{R})$ on the WF basis
                 is expected to have a nearly exponential decay wrt $\mathbf{R}$ when
                 WF are well-localized.
        \item    Using the {\tt bands.x} postprocessing, it is possible to interpolate
                 the band structure using the computed $H(\mathbf{R})$ matrices.
                 If WF are well-behaved, bands are usually reproduced using a small
                 number of $\mathbf{R}$ vectors (since the real space decaying of
                 the Hamiltonian). Otherwise this last assumption is not allowed and
                 the band structure typically shows unphysical oscillations.
        \end{itemize}

\item   {\bf What about the symmetry properties of the WF set ?} \\
        When computing WFs for the occupied manyfold of valence bands,
        the total charge, which is fully symmetric, can be alternatively written as
        the sum of the WF squared moduli.
        WFs retain therefore a direct link to the symmetry properties of the crystal.
        This is no longer guaranteed when we are mixing valence and conduction states
        in the most general way (in the {\tt disentangle.x} run).
        A total lack of symmetry in these cases is not intrinsically the sign of a bug, but
        may be the sign of some unwanted behavior of the localization process. See
        Sec.~\ref{subsec:troubleshoot} in order to try to avoid these results.


\end{itemize}


%%%%%%%%%%%%%%%%%%%%%%%%%%%%%%%%%%%%%%%%%%%
\subsection{Troubleshootting (sort of)}
\label{subsec:troubleshoot}

%
%
\begin{itemize}
\item   {\bf Energy window and {\tt dimwin}} \\
        If the {\tt disentangle.x} code complains when starting the run about
        energy windows or about the variable {\tt dimwin}, it means that the chosen window
        is probably too small.  For each $\mathbf{k}-$point the window
        must contain a number of bands (called {\tt dimwin}) not smaller than
        the number of required Wannier functions ({\tt dimwann}).

\item   {\bf The code complains about $\mathbf{k}$-points} \\
        The $\mathbf{k}$-point checking algorithm has been found to be quite stable,
        and check failures are typically due to errors in calculation setup.
        The most common problem is related to a non-regular Monkhorst-Pack 
        mesh. The code is not yet able to take advantage of space symmetries (and
        neither of time reversal) and therefore DFT calculations should be done using
        a regular grid on the whole Brillouin Zone and not on its irreducible wedge only.
        Please report other failures if any.

\item   {\bf How can I set the starting Wannier centers ? } \\
        The position and the type of starting centers are very important to speed up
        the convergence of both {\tt disentangle.x} and {\tt wannier.x}. They are
        anyway much less effective on the final results of the minimizations,
        which usually do not depend that much on them. Standard positions for
        centers are on the atomic sites (you can use both {\tt "1gauss"} and
        {\tt "atomic"} types) and on the covalent bonds ( using {\tt "1gauss"} type
        with s-symmetry). When using atomic positions, angular channels are usually
        attributed on the basis of the related atomic orbitals.
        When the chemical environment is more complex or you are
        requiring a lot of WFs the choice may be more difficult but probably also
        less effective on the minimization speed. Take a look at the
        test suite to see how simple examples work.

\item   {\bf When reading overlaps or projections the code exits with an error } \\
        The flags {\tt overlaps="from\_file"} and {\tt projections="from\_file"}
        (which are silently activated also in the restart procedure) make the
        code to read overlaps and projections from a datafile of a previous run.
        At the moment 
        this operation is allowed only if the dimensions of these data are the
        same as in the current calculation. Strictly speaking, if you want to use 
        these flags you should not change the energy window, 
        the number of Wannier functions, etc.

\item   {\bf {\tt disentangle.x} does not converge } \\
        This is a quite strange behavior: if it only takes so long, please enlarge
        the maximum number of steps. On the contrary, if it largely oscillates
        leaving no hope for a convergence, please report your input file to the
        developers.

\item   {\bf {\tt disentangle.x} converges, but average spread is large } \\
        This usually means that some bands needed to achieve a better localization
        are out of the energy window which should be enlarged. This operation should
        be done with care since too large windows will lead to small spread on one
        side, but the obtained subspace will acquire components on very high energy
        Bloch states. Usually this must be avoided since we would like to represent
        a physically interesting subspace.
        Looking at the end of the {\tt disentangle.x} output file when setting the
        variable {\tt verbosity="high"} you will be able to see all the projections
        of the Bloch states onto the computed subspace.

\item   {\bf $U^{\mathbf{k}}$ matrices are not unitary } \\
        These failures are detected by {\tt WARNING} keywords in the {\tt disentangle.x}
        output files and lead to calculations stopping in {\tt wannier.x} or
        post-processing. If you are dealing with a very large systems
        (hundreds of atoms) first try to enlarge the check threshold ({\tt unitary\_thr}).
        Values larger than $10^{-6}, 10^{-5}$ (or even of the same order for small systems)
        are usually a sign of numeric instability, probably due to porting issues.
        Please report any failure of this kind.

\item   {\bf Wannier functions does not converge} \\
        The {\tt wannier.x} localization algorithm may take so long to reach the
        convergence. Many large oscillations in the total spread may also happen.
        If conditioned minimization are not used, be confident and let the code
        going on in the iterative minimization (restart or enlarge the maximum number
        of steps). If the minimization takes too long, consider it not converged and
        look at the following point.

\item   {\bf Wannier functions converge, but they are weird } \\
        This is a very common case. The quality of WFs depends so much on the starting
        subspace provided by {\tt disentangle.x}. Try to modify it (avoid high
        energy Bloch state components) properly re-setting the energy window.
        This is anyway a quite drastic measure. Before doing this you can try to
        change the starting centers (in this case you should also re-run
        {\tt disentangle.x} since projections must be re-calculated).

        If you understand that WFs are not well-behaved since they got in strange
        positions you may wonder to use conditioned minimization and penalty functionals.
        See Sec.~\ref{subsubsec:penalty} for the theoretical details of the method.
        In the current implementation, the conditioning tries to move the centers of
        WFs as close as possible to the positions of the starting centers provided
        in the input. Their importance is therefore crucial and bad centers are
        easily detected since the minimization path is completely crazy.
        The use of this conditioning is typically useful when the simulation cell
        includes some vacuum regions.

\item   {\bf Transport calculations have no physical meaning } \\
        Unfortunately the input file for transport calculation (particularly the
        card related to the real space Hamiltonians) must be carefully set.
        When your transport calculations are completely wrong (weird oscillations or
        narrow peaks in the transmittance) it is usually due to some error in
        the input file ({\it e.g.} the indices of the required submatrices).

\item   {\bf Transport calculations have has some sense but are noisy} \\
        Numerical inaccuracies may be present. Typical effects are boring dips
        in the transmittance curves. Make sure that your conductor
        cell is converged wrt the bulk properties of the lead: if not, enlarge the cell.
        Are your principal layer large enough to make their interaction decaying
        just after nearest neighbors ?
        Are the energy levels of your conductor and lead calculation the same ?
        See Sec.~\ref{sec:run} about calculation setup to know how to check it.

\end{itemize}
