%%%%%%%%%%%%%%%%%%%%%%%%%%%%%%%%%%%%%%%%%%%%%%%%%%%%%%%%%%
%  Copyright (C) 2005 WanT group                         %
%  This file is distributed under the terms of the       %
%  GNU General Public License.                           %
%  See the file `License'  in the root directory of      %
%  the present distribution,                             %
%  or http://www.gnu.org/copyleft/gpl.txt                %
%%%%%%%%%%%%%%%%%%%%%%%%%%%%%%%%%%%%%%%%%%%%%%%%%%%%%%%%%%

\thispagestyle{empty}
\section{The test suite}
\label{section:test}
%
\WANT\ package is distributed with a suite of tests which are
intended to check the portability of the code and to show its
features. Along these lines, reference results as well as detailed
explanations are supplied.  A discussion on how to fruitfully use
the test suite is reported in the following.

%%%%%%%%%%%%%%%%%%%%%%%%%%%%%%%%%%%%%%%%%%%
\subsection*{HOW TO set up the environment}
     In order to get started with running tests you should modify the
     {\tt \$TOPDIR/tests/environment.conf} file according to your system.
     You should provide the location of the executables for DFT calculation
     (at present, tests are suited for the \PWSCF code) and the directory
     that will be used for temporary files ({\tt scratch}).
     In the case of parallel machines you are also requested to
     specify some special commands to run the DFT codes within an
     MPI environment.

%%%%%%%%%%%%%%%%%%%%%%%%%%%%%%%%%%%%%%%%%%%
\subsection*{HOW TO run the tests}
     After environment setting, you can directly start running tests. The
     script {\tt run.sh} in {\tt \$TOPDIR/tests/} (type {\tt ./run.sh} to get a
     short manual) manages all operations on the tests.
     The script should be run with the following arguments: \\

     {\tt  ./run.sh -r <what>  [<test\_list>] } \\

     \noindent
     {\tt <test\_list>} is optional and is the list of the tests we want to
     perform. A missing list performs all tests.
     Possible actions ({\tt <what>}) to be performed are: \\

     %
     %
     \begin{tabular}{ll}
{\tt help}     &       print the manual  \\
{\tt info}     &       print detailed info about the implemented tests \\
{\tt all}      &       perform all the calculations \\
{\tt dft}      &       perform DFT calculations only \\
{\tt want}     &       perform \WANT calculations only \\
{\tt check}    &       check results with the reference outputs \\
{\tt clean}    &       delete all output files and the temporary directory \\
\end{tabular}
%
%
\\

     \noindent
     Tests can also be performed
     by cd-ing in the specific test directory and running the local script
     typing {\tt ./run.sh <flag>}.
     The list of possible {\tt <flag>}'s and a brief explanation is printed by
     executing {\tt ./run.sh} (typical flags with obvious meanings are
     {\tt scf, nscf, disentangle, wannier})
     If you type {\tt ./run.sh all}, inside a specific
     test directory, the test will be completely performed.
     Note that some tests contain more than one \WANT{} calculation; each of them
     has its specific {\tt run\_\$suffix.sh} script.

%%%%%%%%%%%%%%%%%%%%%%%%%%%%%%%%%%%%%%%%%%%
\subsection*{HOW TO check the tests}
%
     In each test directory you will find the {\tt Reference} directory
     containing the distributed reference output files.
     Each {\tt run.sh} script accepts the flag {\tt check}: this
     prints a short version of the {\tt diff} between the current output files and
     the reference ones.
     If you need more detail (as in the case of a clear failure), you are advised
     to directly check your output with a simple {\tt diff} command: \\

        {\tt diff myreport.out Reference/myreport.out } \\

     \noindent
     Doing this, remember that you will usually obtain a large amount of data:
     since for instance the phases of the wave-functions are almost
     randomized, the initial steps both in disentanglement and Wannier minimizations
     may be very different.

%%%%%%%%%%%%%%%%%%%%%%%%%%%%%%%%%%%%%%%%%%%
\subsection*{HOW TO understand the tests}
     In each test directory you will find a {\tt README} file describing the
     physics of the test and the results obtained in the calculation.
     Some tricky and subtle problems about the input files
     are eventually raised and discussed.
