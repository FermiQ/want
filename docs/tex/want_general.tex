%%%%%%%%%%%%%%%%%%%%%%%%%%%%%%%%%%%%%%%%%%%%%%%%%%%%%%%%%%
%  Copyright (C) 2005 WanT group                         %
%  This file is distributed under the terms of the       %
%  GNU General Public License.                           %
%  See the file `License'  in the root directory of      %
%  the present distribution,                             %
%  or http://www.gnu.org/copyleft/gpl.txt                %
%%%%%%%%%%%%%%%%%%%%%%%%%%%%%%%%%%%%%%%%%%%%%%%%%%%%%%%%%%

\thispagestyle{empty}
\begin{centering}
{\LARGE \WANT\ Version \WANTVERSION}\\
\end{centering}
\vspace{0.35in}

\noindent {\bf CREDITS.} The development and  maintenance of the
\WANT\ code is promoted by the National Research Center on
nanoStructures and bioSystems at Surfaces (S3) of the Italian
INFM-CNR (http://www.s3.infm.it) and the Physics Department North
Carolina State University (NCSU) (http://ermes.physics.ncsu.edu)
under the coordination of Arrigo Calzolari, Andrea Ferretti  and
Marco
Buongiorno Nardelli.\\

\noindent The present release of the \WANT\ package has been
realized by Andrea Ferretti (S3), Arrigo Calzolari (S3) and Marco
Buongiorno Nardelli (NCSU).

\noindent A list of contributors includes Carlo Cavazzoni
(CINECA), Benedetta Bonferroni (S3) and Nicola Marzari (MIT). \\

\noindent The routines for the calculation of the
maximally-localized Wannier functions were originally written by
Nicola Marzari and David Vanderbilt (\copyright 1997);  Ivo Souza,
Nicola Marzari and David Vanderbilt (\copyright 2002); Arrigo
Calzolari, Nicola Marzari, and Marco Buongiorno Nardelli (\copyright 2003).\\

\noindent The routines for the calculation of the quantum
conductance were originally written by Marco Buongiorno Nardelli
(\copyright 1998); Arrigo Calzolari, Nicola Marzari, and
Marco Buongiorno Nardelli (\copyright 2003).\\
 \vspace{0.25in}

\noindent {\bf GENERAL DESCRIPTION.} \WANT\ is an open-source, GNU
General Public License suite of codes that provides an integrated
approach for the study of coherent electronic transport in
nanostructures. The core methodology combines state-of-the-art
Density Functional Theory (DFT), plane-waves, pseudopotential
calculations with a Green's functions method based on the Landauer
formalism to describe quantum conductance. The essential
connection between the two, and a crucial step in the calculation,
is the use of the maximally-localized Wannier function
representation to introduce naturally the ground-state electronic
structure into the lattice Green's function approach at the basis
of the evaluation of the quantum conductance. Moreover, the
knowledge of Wannier functions allows for a
direct link between the electronic transport properties of the
device with the nature of the chemical bonds, providing insight
into the mechanisms that
govern electron flow at the nanoscale.\\

\noindent The \WANT\ package operates, in principles, as a simple
post-processing of any standard electronic structure code. In its
present version \WANTVERSION\ the user will find a wrapper to run
\WANT\ from the results of a self-consistent calculation done
using the \PWSCF package (\PWSCFURL).\\

\noindent \WANT\ capabilities include: 
- Quantum conductance spectrum for a bulk (infinite, periodic)
system and for a lead-conductor-lead geometry - Density of states
spectrum projected on the conductor region - Centers and spreads of the
maximally-localized Wannier functions of the system.\\

\newpage
\noindent{\bf TERMS OF USE.} Although users are not under any
obligation in the spirit of the GNU General Public Licence, the
developers of \WANT\ would appreciate the acknowledgment of the
effort to produce such codes in the form of the following
reference:\\

\noindent In the text: ``The results of this work have been
obtained using the \WANT\ package.[ref]''\\

\noindent In references: ``[ref] \WANT\ code by A. Calzolari, A.
Ferretti, C. Cavazzoni, N. Marzari and M. Buongiorno Nardelli,
(www.wannier-transport.org). See also: A. Calzolari, N. Marzari,
I. Souza and M. Buongiorno Nardelli, Phys. Rev. B 69, 035108
(2004).''\\
  \vspace{0.25in}

\noindent{\bf DISCLAIMER.} While the developers of \WANT\ make
every effort to deliver a high quality scientific software, we do
not guarantee that our codes are free from defects. Our software
is provided ``as is''. Users are solely responsible for determining
the appropriateness of using this package and assume all risks
associated with the use of it, including but not limited to the
risks of program errors, damage to or loss of data, programs or
equipment, and unavailability or interruption of operations. Due
to the limited human resources involved in the development of this
software package, no support will be given to individual users for
either installation or execution of the codes. Finally, in the
spirit of every open source project, any contribution from
external users is welcome, encouraged and, if appropriate, will be
included in future releases.\\
  \vspace{0.25in}

\noindent {\bf  LICENCE.} All the material included in this
distribution is free software; you can redistribute it and/or
modify it under the terms of the GNU General Public License as
published by the Free Software Foundation; either version 2 of the
License, or (at your option) any later version. These programs are
distributed in the hope that they will be useful, but WITHOUT ANY
WARRANTY; without even the implied warranty of MERCHANTABILITY or
FITNESS FOR A PARTICULAR PURPOSE. See the GNU General Public
License for more details. You should have received a copy of the
GNU General Public License along with this program; if not, write
to the Free Software Foundation, Inc., 675 Mass Ave, Cambridge, MA
02139, USA.
