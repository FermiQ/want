\section{Simulation Parameters}
\label{section:input_par}

\subsection{\PDAC\ input namelists}
\label{section:namelists}

\subsubsection{\OBJ{control} namelist}
The \OBJ{control} namelist contains the parameters that control the program
flow, such as the start mode, the initial and final time, the output printing.

\begin{itemize}
\item
\PDACCONFWDEF{run\_name}{run identificative}{any string}{{\tt pdac\_run\_2d}}
{This string identifies the simulation. Its length must not exceed 80 characters. It must be enclosed by quotes.}

\item
\PDACCONFWDEF{job\_type}{2d or 3d}{{\tt '2d','2D','3d','3D'}}{{\tt '2D'}}
{When 2D is selected, the model equations are simplified by assuming
some simmetry in the physical domain (translation in one cartesian 
direction or cylindrical symmetry)}

\item
\PDACCONFWDEF{restart\_mode}{flag for restart}{{\tt 'from\_scratch'}, {\tt 'restart'}}
{{\tt 'from\_scratch'}}
{If {\tt restart} is selected, initial conditions are set up by using the fields
dumped (in double precision) into the restart file {\tt pdac.res} (see above). 
Otherwise, initial conditions are set up from input data in {\tt pdac.dat}} 

\item
\PDACCONFWDEF{time}{initial time}{double real}{0.0}
{Inital simulation time (in seconds)}

\item
\PDACCONFWDEF{tstop}{end time}{double real}{100.0}
{Time (in seconds) at which the simulation stops}

\item
\PDACCONFWDEF{dt}{time step}{double real}{0.01}
{Time advancement step (in seconds). 
{\tt dt} is constrained by the CFL condition
$dt < C_{max}\frac{\Delta x}{V_{max}}$, where $\Delta x$ is
the size of the computational grid and $V_{max}$ is the maximum
velocity. The maximum CFL number $C_{max}\approx 0.2$ has been 
found empirically.}

\item
\PDACCONFWDEF{lpr}{level of verbosity}{1,2,3}{2}
{increases the level of verbosity in warning and error messages 
in {\tt pdac.log}, {\tt pdac.err}, {\tt pdac.tst} files}

\item
\PDACCONFWDEF{tpr}{time interval for OUTPUT file}{$>$ dt}{1.0}
{OUTPUT files are printed every {\tt tpr} seconds of simulated time}

\item
\PDACCONFWDEF{tdump}{time interval for restart file}{$>$ dt}{20.0}
{restart file is overwritten every {\tt tdump} seconds of simulated time}

\item
\PDACCONFWDEF{max\_seconds}{maximum CPU time}{any real value}{20000.0}
{Maximum duration of a simulation (in seconds). If time exceeds this value
a restart file is written before the simulation is stopped (useful for
scheduling). Default time is set to 6 hours}

\item
\PDACCONFWDEF{nfil}{number of first OUTPUT file}{any positive integer up to 9999}{0}
{OUTPUT files are written with 4 digits extension ({\tt OUTPUT.XXXX}). 
The output time can be recovered as $({\tt XXXX - nfil})* {\tt tpr}$ }

\item
\PDACCONFWDEF{formatted\_output}{flag for OUTPUT format}{T/F}{T}
{Determine the format of output files: T - formatted ascii, 
F - binary 4-bytes }

\end{itemize}

\subsubsection{\OBJ{model} namelist}
The \OBJ{model} namelist is intended for all switches that can be selected
by the user to modify the model equations, either by neglecting some terms
in the transport equations or by using different constitutive equations
and submodels.

\begin{itemize}

\item
\PDACCONFWDEF{irex}{chemical reactions}{0,1}{0}
{The chemical reaction module has not yet been implemented}

\item
\PDACCONFWDEF{gas\_viscosity}{gas diffusive transport}{{\tt T/F}}{{\tt T}}
{T to solve the full set of model equations. F switches the diffusive
transport terms off for the gas phase (viscous and turbulent term in momentum 
equation, thermal diffusivity in enthalpy equation). 
Gas viscosity is computed anyway to include the gas-particle drag and
energy exchange terms.}

\item
\PDACCONFWDEF{part\_viscosity}{particle diffusive transport}{{\tt T/F}}{{\tt T}}
{Switches on/off the diffusive terms in the particle transport equations 
(viscous terms in the momentum equation and thermal conductivity).}

\item
\PDACCONFWDEF{iss}{turbulence model for particles}{0,1}{0}
{1: computes a gas-analogous sub-grid stress (turbulent viscosity)
for particles.}

\item
\PDACCONFWDEF{repulsive\_model}{flag for Coulombic repulsive model}{0,1}{1}
{The Coulombic repulsive model add a contribution to the diagonal part of
the solid viscous stress due to the repulsive interaction of particles 
at high concentrations. This term appears to be important for the solid
equation to be well-posed.}

\item
\PDACCONFWDEF{iturb}{turbulence model for gas}{0,1,2}{1}
{0 - no gas turbulence model; 1 - Smagorinsky sub-grid stress model;
2 - Smagorinsky sgs model with roughness closure at the walls.}

\item
\PDACCONFWDEF{modturbo}{Subgrid-scale model for gas turbulence}{1,2}{1}
{1 - Classical Smagrinsky model; 2 - Dynamic Smagorinsky model}

\item
\PDACCONFWDEF{cmut}{Smagorinsky constant}{usually between 0.1 and 0.4}{0.1}
{The exact value cannot be predicted {\it a priori}. The use of the dynamic
Smagorinsky model (see above) makes the assignment of this constant
non-necessary}

\item
\PDACCONFWDEF{rlim}{Multiphase limit}{very small real value}{$10^{-8}$}
{It is related to minimum particle concentration for which multiphase flow
equations are solved. Below this limit one-phase equations are solved
(often critical for convergence on the cloud margins).}

\item
\PDACCONFWDEF{gravx}{acceleration along x}{any real}{0.0}
{Body-force in x(r)-direction}

\item
\PDACCONFWDEF{gravy}{acceleration along y}{any real}{0.0}
{Body-force in y-direction}

\item
\PDACCONFWDEF{gravz}{acceleration along z}{any real}{-9.81}
{Body-force in z-direction (usually the value of the gravitational acceleration).
Values different from the default value could not be consistent with the 
atmospheric stratification, leading to instabilities. A value of 0.0 
suppresses atmospheric stratification}.

\item
\PDACCONFWDEF{ngas}{number of gas components}{1 to 7}{2}
{Seven gas species are defined, specifically: \\
$ 1) O_2, 2) N_2, 3), CO_2, 4) H_2, 5) H_2O, 6) Air, 7) SO_2$. 
Only gas species that are specified at the inlet or in the
atmosphere are considered by the model. The total number of gas 
species specified by this flag must be therefore consistent with 
input conditions, otherwise the program stops.}

\item
\PDACCONFWDEF{density\_specified}{flag for specified flow conditions}{T/F}{F}
{Setting this flag to T allows to specify gas density at the inlet instead of
temperature. Gas temperature is then computed by using the perfect gas thermal
equation of state.}

\end{itemize}

\subsubsection{\OBJ{mesh} namelist}
Here all parameters concerning the spatial discretization of the computational
domain must be selected. In \PDAC\ you are constrained to discretization 
on a rectilinear (non-)uniform mesh. Cylindrical coordinates can be
selected only in 2D.

\begin{itemize}
\item
\PDACCONFWDEF{nx}{Number of cells in x(r)-direction}{up to 512}{100}
{When 2D cylindrical coordinates are selected, ``x'' is used instead 
 of ``r''. This number includes the boundary ghost cells. The maximum number can be 
 increased by modifying the max\_size parameter in the ``dimensions'' module}

\item
\PDACCONFWDEF{ny}{Number of cells in y-direction}{up to 512}{1}
{Not used in 2D. It includes the boundary ghost cells.}

\item
\PDACCONFWDEF{nz}{Number of cells in z-direction}{up to 512}{100}
{z is the second space coordinate in 2D. It includes the boundary ghost cells.}

\item
\PDACCONFWDEF{itc}{flag for cylindrical coordinates}{0,1}{0}
{In 2D simulations, itc=1 sets the grid and the coordinates to cylindrical,
by modifying the discretized equations.}

\item
\PDACCONFWDEF{iuni}{flag for uniform mesh}{0,1}{0}
{0 - non uniform mesh; 1 - uniform mesh, takes {\tt dx0, dy0, dz0} as the
cells size in the two/three coordinate directions.}

\item
\PDACCONFWDEF{dx0}{cell sizes in x(r)-direction}{any real}{10.D0}
{Uniform cell sizes along x(r)}

\item
\PDACCONFWDEF{dy0}{cell sizes in y-direction}{any real}{10.D0}
{Uniform cell sizes along y}

\item
\PDACCONFWDEF{dz0}{cell sizes in z-direction}{any real}{10.D0}
{Uniform cell sizes along z}

\item
\PDACCONFWDEF{origin\_x}{origin of the cartesian space}{any real}{0.0}
{Origin of the x-axis}

\item
\PDACCONFWDEF{origin\_y}{origin of the cartesian space}{any real}{0.0}
{Origin of the y-axis}

\item
\PDACCONFWDEF{origin\_z}{origin of the cartesian space}{any real}{0.0}
{Origin of the z-axis}

\item
\PDACCONFWDEF{mesh\_partition}{domain decomposition criterion}{1,2,3}{1}
{1 - Layers decomposition, 2 - 2D Blocks decomposition, 3 - 3D Blocks decomposition}

\end{itemize}

\subsubsection{\OBJ{particles} namelist}
In this namelist the number of solid phases and the physical properties of the
particles forming each phase are specified. Note that the order in which 
particle classes are specified must be the order of the solid-phases array 
storage, in order to be consistent with the initial conditions assignement.

\begin{itemize}

\item
\PDACCONFWDEF{nsolid}{number of solid phases}{up to 10}{2}
{The number of solid phases considered in the multiphase equations.
The maximum number can be increased by modifying the max\_nsolid parameter
in the ``dimensions'' module.}

\item
\PDACCONFWDEF{diameter}{effective diameter particles (in microns)}{real}{100 microns}
{The model works well for particle diameters smaller than few millimeters}

\item
\PDACCONFWDEF{density}{microscopic particle density}{real}{2700 Kg/m$^3$}
{Depends on materials and porosity}

\item
\PDACCONFWDEF{sphericity}{particle sphericity}{0.5 to 1.0}{1.0}
{Partially accounts for particle shapes}

\item
\PDACCONFWDEF{viscosity}{Empirical viscosity coefficient}{0.5 to 2.0}{0.5[Pa s]}
{Largest values apply to coarse particles}

\item
\PDACCONFWDEF{specific\_heat}{solid specific heat}{real}{1.2D3 [J/(K Kg)]}
{Depends on materials}

\item
\PDACCONFWDEF{thermal\_conductivity}{solid thermal conductivity for Fourier law}{real}{2.D0}
{Depends on materials}

\end{itemize}

\subsubsection{\OBJ{numeric} namelist}
By modifying these parameters, you can modify the way \PDAC\ solves the 
model equations. We recommend non-expert users to avoid modifying the
default values.

\begin{itemize}

\item
\PDACCONFWDEF{rungekut}{order of Runge-Kutta explicit integration}{1,2,3}{1}
{The low-storage Runge-Kutta algorithm is used for explicit time integration.
The coefficients used in the RK integration are well suited only up to the
third-order. High-order temporal integration is recommended for 
convergence and stability when high-order spatial discretization schemes 
are used.}

\item
\PDACCONFWDEF{beta}{degree of upwinding}{0.0 to 1.0}{0.25}
{When High Order spatial discretization schemes are used and the MUSCL 
beta-scheme (both limited or unlimited) is selected, the degree of upwinding
can be chosen: \OBJ{beta}=0.0 corresponds to a completely centered schemes:
\OBJ{beta}=1.0 corresponds to a completely upwinded method. The accuracy depends
on the exact value: for \OBJ{beta}=0.25 and \OBJ{beta}=0.33 the formal third-order
is achieved.}

\item
\PDACCONFWDEF{muscl}{flag for MUSCL technique}{0,1}{0}
{The MUSCL technique extends the first-order discretization
scheme to high-orders by reconstructing the flux profile more accurately.
0 - uses first-order upwind; 1 - uses MUSCL technique.}

\item
\PDACCONFWDEF{lim\_type}{limiter type}{0,1,2,3,4}{0}
{Select the type of MUSCL reconstruction and the limiter:\\
0 - beta scheme unlimited; 1 - Van Leer limiter; 2 - Minmod limiter;
3 - Superbee limiter; 4 - beta limited.}

\item
\PDACCONFWDEF{inmax}{maximum number of inner iterations}{integer (small)}{8}
{The exact value does not affect strongly the convergence but can slow down
the simulation. Optimal value is set by default.}

\item
\PDACCONFWDEF{maxout}{maximum number of Gauss-Siedel iterations for convergence}{integer (large)}{5000}
{Convergence is usually reached within less than 100 iterations. If maxout is reached the code CRASH!es}

\item
\PDACCONFWDEF{omega}{over/under-relaxation parameter}{0.0 to 2.0}{1.0}
{Values between 0.0 and 1.0 under-relax the iterative procedure, whereas
values above 1.0 overrelax.}

\item
\PDACCONFWDEF{implicit\_fluxes}{flag for implicit computation of fluxes}{{\tt T/F}}{F}
{By selecting T, convective fluxes are computed (at the first or second-order, 
depending on flag \OBJ{muscl}) within the iterative solver. This modification
significantly slows down the computation but could relax the CFL constraint
so that larger time-steps can be used.}

\item
\PDACCONFWDEF{implicit\_enthalpy}{flag for implicit computation of enthalpies}
{{\tt T/F}}{F}
{Selecting the implicit solution of enthalpies enhances the level of coupling
between the momentum and enthalpy equations. Nevertheless the convective part 
of the enthalpy equations can be left out from the iterative solver
by opportunely selecting the flag \OBJ{implicit\_fluxes}. The performance loss
in any case is significant.}
\end{itemize}

\subsection{\PDAC\ input cards}
\label{section:cards}
\subsubsection{\OBJ{'ROUGHNESS'} card}

The \OBJ{'ROUGHNESS'} card is only valid for 2D 
simulations (in 3D it will be replaced by the roughness matrix). In this
card three elements must be specified:
\begin{itemize}
\item
\PDACCONF{ir}{number of rough zones}{1,2}
{The code allows the specification of one or two roughness regions, characterized
by different roughness lengths.}

\item
\PDACCONF{zrough(:)}{roughness length}{real}
{Specify one roughness length for each rough region. Typical values for ground
ranges from few centimeters to some metres.}

\item
\PDACCONF{roucha}{distance of roughness change}{real}
{The distance from the left boundary at which the roughness changes.}

\end{itemize}

\subsubsection{\OBJ{'MESH'} card}

The \OBJ{'MESH'} card contains the two (in 2D) or three (in 3D) arrays of cell 
sizes specified for a non-uniform rectilinear mesh. 

\begin{itemize}
\item
\PDACCONF{dx(1:nx)}{Array of the cell sizes in x(r)-direction}{real}
{Array of the cell sizes in x(r)-direction.}

\item
\PDACCONF{dy(1:ny)}{Array of the cell sizes in y-direction}{real}
{Array of the cell sizes in y-direction.
Not present in 2D.}

\item
\PDACCONF{dz(1:nz)}{Array of the cell sizes in z-direction}{real}
{Array of the cell sizes in z-direction.}
\end{itemize}

Array elements can be separated by commas, tabs, blanks or lay on different
lines.  When uniform mesh is selected by {\tt iuni=1} parameter in the 
{\tt mesh} namelist, the card can be empty (but the card 
name must always be present in the input file).

\subsubsection{\OBJ{'FIXED\_FLOWS'} card}

The \OBJ{'FIXED\_FLOWS'} card is designed to specify boundary conditions (b.cs.).
Its name is due to the possibility to assign within this card any region with
specified flow conditions (e.g. for inlet flow). B.cs. are imposed in 
mesh region determined by rectangular blocks. The number of blocks is the
first parameter specified in the card

\begin{itemize}
\item
\PDACCONF{number\_of\_block}{number of blocks}{integer}
{The number of blocks used to specify b.c.}
\end{itemize}

Each block is characterized
by a flag that specifies the kind of b.c., and by two (in 2D) or three (in 3D)
couples of integer that specify the first and the last cell of the block in 
the two (three) directions. Each block is therefore identified by 5 (in 2D)
or 7 (in 3D) integers:\\

\begin{itemize}
\item
\PDACCONF{block\_type}{type of b.c.}{1 to 7}
{1 - fluid cells: all equations are solved; 2 - free-slip;
 3 - no-slip; 4 - free in/outflow; 5 - specified flow (inlet); 
 6 - specified pressure; 7 - specified input profile (only 2D).}

\item
\PDACCONF{block\_bounds}{limits of blocks}{integer}
{two or three couples of integers specifying 
$ i_{min}, i_{max}, (j_{min}, j_{max}), k_{min}, k_{max}$ }

\end{itemize}

Blocks are used also to specify ``blocking cells'' (e.g. 
the volcano topography). When the block type is equal to 5 (specified
flow), inlet conditions must be specified as follows:\\

First line:

\begin{itemize}
\item
\PDACCONF{fixed\_vgas\_x}{gas velocity component along x}{real}
{Inlet gas velocity along x(r)-direction. Note that CFL condition (see above for \OBJ{dt}) 
must be satisfied for every component.}

\item
\PDACCONF{fixed\_vgas\_y}{gas velocity component along y}{real}
{Inlet gas velocity along x(r)-direction (not present in 2D!).}

\item
\PDACCONF{fixed\_vgas\_z}{gas velocity component along z}{real}
{Inlet gas velocity along z-direction.}

\item
\PDACCONF{fixed\_pressure}{pressure at inlet}{real}
{The thermodynamic pressure of the gas phase.}

\item
\PDACCONF{fixed\_gaseps}{inlet gas volumetric fraction}{ $\le 1.0$}
{This value must be consistent with the solid phases volumetric fractions.
The closure relation imposes that the sum equals one.}

\item
\PDACCONF{fixed\_gastemp}{inlet gas temperature}{ a positive real }
{No constraint is imposed on gas temperature, but the range of validity
of the equation of state must be check.}
\end{itemize}

Further {\tt nsolid} lines:
\begin{itemize}
\item
\PDACCONF{fixed\_vpart\_x}{particle velocity along x}{real}
{Inlet particle velocity along x(r)-direction.
Note that CFL condition (see above for \OBJ{tt}) 
must be satisfied for every component.}

\item
\PDACCONF{fixed\_vpart\_y}{particle velocity along y}{real}
{Inlet particle velocity along y-direction(not present in 2D!).}

\item
\PDACCONF{fixed\_vpart\_z}{particle velocity along z}{real}
{Inlet particle velocity along z-direction.}

\item
\PDACCONF{fixed\_parteps}{particle volumetric fraction}{ $\le 1.0$}
{Must satisfy closure relation of total volumetric fraction}

\item
\PDACCONF{fixed\_parttemp}{particle temperature}{ a positive real}
{Particle temperature is not constrained by a thermal equation of state}

\end{itemize}

Last line:

\begin{itemize}

\item
\PDACCONF{fixed\_gasconc(1:7)}{concentration of each gas species at inlet}{$\le 1.0$}
{The mass fraction (concentration) of each of the seven gas species must be
specified. The closure relation for the total mass fraction must be satisfied.
If this constrain is not satisfied, the {\tt default\_gas} mass fraction is
automatically corrected.}

\end{itemize}

\subsubsection{\OBJ{'INITIAL\_CONDITIONS'} card}

The \OBJ{'INITIAL\_CONDITIONS'} card specifies the ambient conditions at 
the beginning of the simulation in all the computational cells of
type 1 (fluid cells).

First line:

\begin{itemize}
\item
\PDACCONF{initial\_vgas\_x}{gas velocity component along x}{real}
{Initial gas velocity along x(r)-direction. Can be used for cross-winds. 
Please note that the CFL condition (see above for \OBJ{dt}) 
must be satisfied for every component.}

\item
\PDACCONF{initial\_vgas\_y}{gas velocity component along y}{real}
{Inlet gas velocity along y-direction (not present in 2D!).}

\item
\PDACCONF{initial\_vgas\_z}{gas velocity component along z}{real}
{Inlet gas velocity along z-direction.}

\item
\PDACCONF{initial\_pressure}{pressure at inlet}{real}
{This corresponds to the thermodynamic ambient pressure (if gravity is switched off) 
or the pressure at the ground (if atmosphere is stratified, values different from 1 atmosphere
lead to a WARNING message).}

\item
\PDACCONF{initial\_void\_fraction}{atmospheric gas volumetric fraction}
{ $\le 1.0$}
{Solid phases volumetric fractions are computed to satisfy the closure relation}

\item
\PDACCONF{initial\_temperature}{inlet gas temperature}{ a positive real }
{This corresponds to the ambient temperature (if gravity is switched off) 
or the temperature at the ground (if atmosphere is stratified, values different from 288.15K
lead to a WARNING message).
No constraint is imposed on gas temperature, but the range of validity
of the equation of state must be check. Particles are supposed to be in thermal
equilibrium with gas.}
\end{itemize}

Further {\tt nsolid} lines:
\begin{itemize}
\item
\PDACCONF{initial\_vpart\_x}{particle velocity along x}{real}
{Initial particle velocity along x(r)-direction.
Note that CFL condition (see above for \OBJ{dt}) 
must be satisfied for every component.}

\item
\PDACCONF{initial\_vpart\_y}{particle velocity along y}{real}
{Initial particle velocity along y-direction (not present in 2D!).}

\item
\PDACCONF{initial\_vpart\_z}{particle velocity along z}{real}
{Initial particle velocity along z-direction.}

\end{itemize}

Last line:

\begin{itemize}

\item
\PDACCONF{initial\_gasconc(1:7)}{concentration of each gas species in atmosphere}{$\le 1.0$}
{The mass fraction (concentration) of each of the seven gas species must be
specified. The closure relation for the total mass fraction must be satisfied.
If this constrain is not satisfied, the {\tt default\_gas} mass fraction is
automatically corrected.}

\end{itemize}
