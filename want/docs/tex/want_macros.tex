%
% generally useful macros
%
\newcommand{\htmlimage}[1] {}
\newcommand{\REFAND} {\&}
\newcommand{\ETALNP}{\mbox{\it et al}}
\newcommand{\ETAL}{\mbox{\ETALNP{\it.}}}
\newcommand{\eqnref}[1] {\mbox{eq (\ref{#1})}}
\newcommand{\mycite}[2] {\cite{#2}}

\newcommand{\PDAC} {PDAC}
\newcommand{\PDACNAME} {Pyroclastic Dispersal Analysis Code}
%\newcommand{\PDACDATE} {\today}
\newcommand{\PDACDATE} {January 2004}
\newcommand{\RMAUTHORS} {
 A.~Neri, T.~Esposti Ongaro, C.~Cavazzoni, G.~Erbacci and G.~Macedonio }
\newcommand{\UGAUTHORS} {
T.~Esposti Ongaro, C.~Cavazzoni and A.~Neri }
\newcommand{\PDACVERSION} {2.0}
\newcommand{\PDACNOTES} {\sc (Restricted Draft: Do Not Release or Cite)}
\newcommand{\PDACADDRESS} {\underline{pdac@pi.ingv.it}}
\newcommand{\PDACURL} {\underline{http://www.pi.ingv.it/PDAC}}
% title of PDAC paper
\newcommand{\PDACPAPER} {PDAC: A transient, multiphase flow code
for the 3D simulation of pyroclastic dispersal dynamics}


%
% macros for style conventions when describing the program.
%

% name of class or object in program
\newcommand{\OBJ}[1] {{\tt#1}}

% name of file
\newcommand{\FIL}[1] {{\it #1}}

% name of directory
\newcommand{\DIR}[1] {$<${\it #1}$>$}

% function arguments
\newcommand{\FA}[2] {{\rm{\bf#1}\ {\it#2}}}

% global function name
\newcommand{\FN}[3] {{\rm\bf#1}\ {\tt #2(}#3{\tt)}}

% class member function name
\newcommand{\FNO}[4] {{\rm\bf#2}\ \OBJ{#1::}{\tt#3(}#4{\tt)}}

% list item, for optional components, parameters, etc.
\newcommand{\TTLISTITEM}[1] {\item {\tt #1} \\}
\newcommand{\RMLISTITEM}[1] {\item {\rm #1} \\}
\newcommand{\BOLDLISTITEM}[1] {\item {\bf #1} \\}
\newcommand{\EMLISTITEM}[1] {\item {\em #1} \\}
\newcommand{\LISTITEM}[1] {\RMLISTITEM{#1}}

%
% other generally useful macros
%

\newcommand{\UGTITLE} {User's Guide}
\newcommand{\RMTITLE} {Reference Manual}
\newcommand{\UGDESC} {
This {\em\UGTITLE\/} describes how to run and use the 
various features of the \PDACNAME\ \PDAC.  
This guide includes a description of the capabilities of the program, how 
to use these capabilities, the necessary input files and 
formats, and how to run the program on both uniprocessor 
and parallel machines.}

\newcommand{\RMDESC} {
The \PDAC\ (\PDACNAME)\ {\em\RMTITLE\/} is intended for the user who wants
to overview the model equations and the numerical technique
or for the programmer who wants to 
cooperate to the \PDAC\ development. The manual describes how the model 
equations are discretized and solved, provides the explicative list of 
all variables used in the numerical code and gives a brief description of the 
parallelization strategy, so that the advanced user can optimize the
code efficiency on several computational platforms.}

\newcommand{\PG}{\PDAC\ {\it Programmer's Guide\/}}
\newcommand{\UG}{\PDAC\ {\it User's Guide\/}}
\newcommand{\RM}{\PDAC\ {\it Reference Manual\/}}
\newcommand{\prettypar}{

\smallskip

}

\newcommand{\eg}{{\it e.g.\/}}
\newcommand{\ie}{{\it i.e.\/}}

\newcommand{\KEY}[1]{{\tt #1}}
\newcommand{\IKEY}[1]{{\tt #1\index{#1 psfgen command}}}
\newcommand{\OKEY}[1]{$[${\tt #1}$]$}
\newcommand{\ARG}[1]{$<${\em #1}$>$}
\newcommand{\OARG}[1]{$[${\em #1}$]$}
\newcommand{\ARGDEF}[2]{$<${\em #1}$>$: #2}
\newcommand{\KEYDEF}[2]{{\tt #1}: #2}
\newcommand{\COMMAND}[4]{%
  #1 \\ {\bf Purpose:} #2 \\ {\bf Arguments:} #3 \\ {\bf Context:} #4 }

\newcommand{\icommand}[1]{#1\index{#1 command}}

\newcommand{\PDACCONF}[4]{%
%  \addcontentsline{toc}{subparagraph}{#1}%
  {\bf \tt #1 } $<$ #2 $>$ \\%
  \index{#1 parameter}
  {\bf Acceptable Values: } #3 \\%
  {\bf Description: } #4%
}

\newcommand{\PDACCONFWDEF}[5]{%
%  \addcontentsline{toc}{subparagraph}{#1}%
  {\bf \tt #1 } $<$ #2 $>$ \\%
  \index{#1 parameter}
  {\bf Acceptable Values: } #3 \\%
  {\bf Default Value: } #4 \\%
  {\bf Description: } #5%
}
