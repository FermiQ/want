%%%%%%%%%%%%%%%%%%%%%%%%%%%%%%%%%%%%%%%%%%%%%%%%%%%%%%%%%%
%  Copyright (C) 2005 WanT group                         %
%  This file is distributed under the terms of the       %
%  GNU General Public License.                           %
%  See the file `License'  in the root directory of      %
%  the present distribution,                             %
%  or http://www.gnu.org/copyleft/gpl.txt                %
%%%%%%%%%%%%%%%%%%%%%%%%%%%%%%%%%%%%%%%%%%%%%%%%%%%%%%%%%%

\thispagestyle{empty}
\section{How to prepare an input file}\label{section:input}

\noindent According to the methodological scheme of Section
\ref{section:run}, it is necessary to use separate input files at
the different step of the \WANT\ procedure.\\

\noindent Input files organized  as several {\bf NAMELIST},
followed by other fields introduced by {\bf CARDS}. Namelist are
defined from the  flag "\&NAMELIST'' at the beginning to the
"$/$'' bar at the end. The order of variables within a namelist is
arbitrary. Most variables have default value. If a variable is not
explicitly defined in the input file, its default value is
assumed.

\noindent In the following we report the list and the description
of the parameters for each required input file.

\subsection{Input for DFT-PW calculations}
\noindent {\bf Step 1. i-ii: pw.x}\\
\noindent \WANT\ is currently interfaced with Pwscf code. For the
description of the input for steps 1-2 (Sec. \ref{section:run})
and for further details see the Pwscf manual at www.pwscf.org.\\

\noindent {\bf Step 1. iii: pw\_export.x}\\
\noindent Namelists allowed : \&inputpp\\
\noindent Input file layout
\begin{description}
  \item \&inputpp
  \item ...
  \item $/$
\end{description}
\begin{centering}
\rule{2.5in}{0.01in} List of variables \rule{2.5in}{0.01in}
\end{centering}\\

\newdimen\descindent \descindent = 8pc
{\noindent \leftskip = \descindent \parskip = .5\baselineskip
\llap{\hbox to \descindent{\bf prefix\hfil}}%
STRING \\ the first part of the name of all the file written by
the code\\ DEFAULT = mandatory \par
\noindent\llap{\hbox to \descindent{\bf outdir\hfil}}%
STRING \\ the scratch directory where the massive data file will
be written\\ DEFAULT = "$./$" \par
\noindent\llap{\hbox to \descindent{\bf pseudo\_dir\hfil}}%
STRING \\directory containing pseudopotential (PP) files\\
DEFAULT = "$./$" \par
\noindent\llap{\hbox to \descindent{\bf psfile(i)\hfil}}%
STRING \\ files containing i-th PP, where i=1, ntype \\ PP
numbering must follow the ordering defined in the input of pw.x
\par
\noindent\llap{\hbox to \descindent{\bf single\_file\hfil}}%
LOGICAL \\ XXXXXXX aggiungere!!!\\ DEFAULT = ".FALSE." \par
\noindent\llap{\hbox to \descindent{\bf ascii\hfil}}%
LOGICAL \\ XXXXXXX aggiungere!!!\\ DEFAULT = ".FALSE." \par}\bigskip

\subsection{Input for Wannier function calculations}
\noindent {\bf Step 2. a-b: disentangle.x wannier.x}\\
\noindent Both codes for the WF calculation (disentangle.x and wannier.x) uses
the same input file.\\

\noindent Namelists allowed : \&CONTROL, \&SUBSPACE, \&LOCALIZATION\\
\noindent Cards allowed : WANNIER\_CENTERS\\
\noindent Input file layout
\begin{description}
  \item \&CONTROL
  \item ...
  \item $/$
  \item
  \item \&SUBSPACE
  \item ...
  \item $/$
  \item
  \item \&LOCALIZATION
  \item ...
  \item $/$
  \item
  \item WANNIER\_CENTERS ( "crystal" $\mid$ "angstrom" $\mid$ "bohr" )
  \item $<$type1$> \qquad    <$specific\_fmt$>$
  \item ...
  \item $<$typeN$> \qquad    <$specific\_fmt$>$
\end{description}

\begin{centering}
\rule{2.2in}{0.01in} NAMELIST \&CONTROL \rule{2.2in}{0.01in}
\end{centering}\\

\newdimen\descindent \descindent = 8pc
{\noindent \leftskip = \descindent \parskip = .5\baselineskip
\llap{\hbox to \descindent{\bf prefix\hfil}}%
STRING \\ the first part of the name of all the file written by
the code\\ DEFAULT = mandatory \par

\noindent\llap{\hbox to \descindent{\bf postfix\hfil}}%
STRING \\ the tail of the names of the above mentioned files (useful e.g. to
distinguish among different calculations having a common part)\\ DEFAULT = "" \par

\noindent\llap{\hbox to \descindent{\bf work\_dir\hfil}}%
STRING \\ the scratch directory where the massive data file will be written)\\ DEFAULT = "$./$" \par

\noindent\llap{\hbox to \descindent{\bf title\hfil}}%
STRING \\ the title of the calculation)\\ DEFAULT = "Wannier Transport Calculation" \par

\noindent\llap{\hbox to \descindent{\bf restart\_mode\hfil}}%
STRING \\ ( "from\_scratch" $\mid$ "restart" ) \\define whether to restart a previous calculation;
at the moment the "restart" choice implies to overwrite the input variables OVERLAPS, PROJECTIONS,
START\_MODE\_DIS and START\_MODE\_WAN , with the value "from\_file" (see below
for thier meanings)\\ DEFAULT = "from\_scratch" \par

\noindent\llap{\hbox to \descindent{\bf verbosity\hfil}}%
STRING \\ ( "low" $\mid$ "medium" $\mid$ "high" )
\\the level of detail of the code output\\ DEFAULT = "medium" \par

\noindent\llap{\hbox to \descindent{\bf overlaps\hfil}}%
STRING \\ ( "from\_scratch" $\mid$ "from\_file" )\\
determine how to get overlap integrals:\\
"from\_scratch":  overlaps are calculated from wfcs\\
"from\_file":     overlaps are read from a previous data file.
In this second case dimensions should be consistent\\ DEFAULT = "from\_scratch" \par

\noindent\llap{\hbox to \descindent{\bf projections\hfil}}%
STRING \\ ( "from\_scratch" $\mid$ "from\_file" )\\
determine how to get projections integrals:\\
Meaning as before\\ DEFAULT = "from\_scratch" \par

\noindent\llap{\hbox to \descindent{\bf assume\_ncpp\hfil}}%
LOGICAL \\ if .TRUE. avoids the reading of pseudopotential files
assuming that the DFT calculation has been performed within norm-conserving
pseudopotentials (no knowledge of them is required in the \WANT\ calc)\\ DEFAULT = ".FALSE." \par

\noindent\llap{\hbox to \descindent{\bf unitary\_thr\hfil}}%
REAL \\ threshold for the check of matrix unitary\\ DEFAULT = 1.0d-8 \par
}\bigskip

\begin{centering}
\rule{2.2in}{0.01in} NAMELIST \&SUBSPACE \rule{2.2in}{0.01in}
\end{centering}\\

\newdimen\descindent \descindent = 8pc
{\noindent \leftskip = \descindent \parskip = .5\baselineskip
\llap{\hbox to \descindent{\bf dimwann\hfil}}%
INTEGER \\ the number of wannier functions, i.e. the dimension of the wannier subspace
\\ DEFAULT = mandatory \par

\noindent\llap{\hbox to \descindent{\bf win\_min\hfil}}%
REAL \\ the lower limit [eV] of the energy window containing the states
forming the starting subspace for Wannier functions)\\ DEFAULT = 0.0 \par

\noindent\llap{\hbox to \descindent{\bf win\_max\hfil}}%
REAL \\ the upper limit [eV] of the energy window described above\\ DEFAULT = 0.0 \par

\noindent\llap{\hbox to \descindent{\bf froz\_min\hfil}}%
REAL \\ the lower limit [eV] of the energy window containing 'frozen' states
which will not enter the calculation of WFs\\ DEFAULT = -2.0000 \par

\noindent\llap{\hbox to \descindent{\bf froz\_max\hfil}}%
REAL \\ upper limit [eV] of the frozen window described above\\ DEFAULT = -1.0000 \par

\noindent\llap{\hbox to \descindent{\bf alpha\_dis\hfil}}%
REAL \\ mixing parameter for the disentangle iterative procedure\\ DEFAULT = 0.5 \par

\noindent\llap{\hbox to \descindent{\bf maxiter\_dis\hfil}}%
INTEGER \\  maximum number of iterations during the disentangle procedure\\ DEFAULT = 1000 \par

\noindent\llap{\hbox to \descindent{\bf nprint\_dis\hfil}}%
INTEGER \\  every nprint\_dis iterations in disentangle minimization write to stdout\\ DEFAULT = 10 \par

\noindent\llap{\hbox to \descindent{\bf nsave\_dis\hfil}}%
INTEGER \\  every nsave\_dis iterations save subspace data to disk\\ DEFAULT = 10 \par

\noindent\llap{\hbox to \descindent{\bf use\_blimit\hfil}}%
LOGICAL \\   if .TRUE., b vectors are set to zero when calculation overlap augmentations.
This essentially means we are doing a sort of thermodynamic limit
even if this is not consistent with the actual kpt grid. The .TRUE. value
should be considered for debug purposes\\ DEFAULT = ".FALSE." \par

\noindent\llap{\hbox to \descindent{\bf disentangle\_thr\hfil}}%
REAL \\  threshold for convergence of the iterative disentangle procedure\\ DEFAULT = 1.0d-8 \par

\noindent\llap{\hbox to \descindent{\bf subspace\_init\hfil}}%
STRING \\  ( "randomized" $\mid$ "lower\_states" $\mid$ "upper\_states" $\mid$ \\
"center\_projections" $\mid$ "from\_file" )\\
Determine how the trial subspace is chosen\\
"randomized"   : random starting point is chosen\\
"lower\_states" : the lower DIMWANN bands from DFT calculation are
                 used to define the subspace\\
"upper\_states" : the upper DIMWANN bands from DFT calculation are
                 used to define the subspace\\
"center\_projections" : a subspace is extracted from the DFT bands
                 by means of a projections on the given WANNIER\_TRIAL\_CENTERS
                 (see the section WANNIER\_CENTERS)\\
"from\_file" : subspace initialization is read from an existing data file;
                 this is the choice used during restart\\
DEFAULT = "center\_projections" \par

\noindent\llap{\hbox to \descindent{\bf spin\_component\hfil}}%
STRING \\  ( "up" $\mid$ "down" $\mid$ "none")\\
defines whether the calculation is spin polarized and if the case
which spin component is to be treated\\ DEFAULT = "none" \par
}\bigskip

\begin{centering}
\rule{2.2in}{0.01in} NAMELIST \&LOCALIZATION \rule{2.2in}{0.01in}
\end{centering}\\

\newdimen\descindent \descindent = 8pc
{\noindent \leftskip = \descindent \parskip = .5\baselineskip
\llap{\hbox to \descindent{\bf wannier\_thr\hfil}}%
REAL \\ threshold for convergence of the iterative wannier minimization
\\ DEFAULT = 1.0d-6 \par

\noindent\llap{\hbox to \descindent{\bf aplha0\_wan\hfil}}%
REAL \\ mixing parameter during the first CG part of the wannier minimization\\ DEFAULT = 0.5 \par

\noindent\llap{\hbox to \descindent{\bf aplha1\_wan\hfil}}%
REAL \\ mixing parameter during the second part of the wannier minimization\\ DEFAULT = 0.5 \par

\noindent\llap{\hbox to \descindent{\bf maxiter0\_wan\hfil}}%
INTEGER \\ maximum number of iterations for the first CG part of the wannier minimization\\ DEFAULT = 0.5 \par

\noindent\llap{\hbox to \descindent{\bf maxiter1\_wan\hfil}}%
INTEGER \\ maximum number of iterations for second part of the wannier minimization\\ DEFAULT = 0.5 \par

\noindent\llap{\hbox to \descindent{\bf nsave\_wan\hfil}}%
INTEGER \\ every nsave\_dis iterations save subspace data to disk\\ DEFAULT = 10 \par

\noindent\llap{\hbox to \descindent{\bf ncg\hfil}}%
INTEGER \\ every ncg iterations in the second minimization part, do a CG minimization\\ DEFAULT = 3 \par

\noindent\llap{\hbox to \descindent{\bf localization\_init\hfil}}%
STRING \\  ( "no\_guess" $\mid$ "randomized" $\mid$ "center\_projections" $\mid$ "from\_file" )\\
 Determine how the wannier localization is started\\
"no\_guess" : disentangle states are used as starting point
                 without any further localization guess\\
"randomized" : a random rotation is applied to the states found by
                 the disentangle procedure\\
"center\_projections" : a subspace is extracted from the DFT bands
                 by means of a projections on the given WANNIER\_TRIAL\_CENTERS
                 (see the section WANNIER\_CENTERS)\\
"from\_file" : subspace initialization is read from an existing data file;
                 this is the choice used during restart \\
DEFAULT = "center\_projections" \par

\noindent\llap{\hbox to \descindent{\bf ordering\_mode\hfil}}%
STRING \\ ( "none" $\mid$ "spatial" $\mid$ "spread" $\mid$ "complete" )
specifies whether to order the computed Wannier functions and
              which ordering criterion adopt\\
"none":      no ordering is performed\\
"spatial":   ordering based on WF center positions\\
"spread":    ordering based on WF increasing spreads\\
"complete":  SPATIAL + SPREAD for WF with the same centers\\
DEFAULT = "none" \par

\noindent\llap{\hbox to \descindent{\bf a\_condmin\hfil}}%
REAL \\ the amplitude of the conditioned minimization functional. If set to zero
              ordinary minimization is performed\\ DEFAULT = 0.0 \par

\noindent\llap{\hbox to \descindent{\bf niter\_condmin\hfil}}%
INTEGER \\ the number of steps for which minimization is conditioned.\\
DEFAULT = $\begin{array}{ll}
           \textrm{maxiter0\_wan + maxiter1\_wan}   & \textrm{(if a\_condmin} \neq 0.0)\\
           0                                 & \textrm{(otherwise)}
           \end{array} $ \par

\noindent\llap{\hbox to \descindent{\bf dump\_condmin\hfil}}%
REAL \\ the dumping factor for a\_condmin during the conditioned minimization.\\
              If the dumping factor is specified, after niter\_condmin iterations a\_condmin
              is dumped according to
                 a\_condmin = a\_condmin $*$ dump\_condmin
              at each iteration\\
DEFAULT = 0.0 \par
}
\bigskip

\begin{centering}
\rule{2.0in}{0.01in} CARD WANNIER\_CENTERS \rule{2.0in}{0.01in}
\end{centering}\\

\noindent {\bf WANNIER\_CENTERS}
( "crystal" $\mid$ "angstrom" $\mid$ "bohr" )\\

\noindent Aside the tag WANNIER\_CENTERS, units for positions maybe specified:\\
$\begin{array}{ll}
\textrm{"crystal"}  & \textrm{: relative coordinates on the basis of a1,a2,a3 lattice vector (default)}\\
\textrm{"bohr"}     & \textrm{: cartesian coordinates in bohr}\\
\textrm{"angstrom"} & \textrm{: cartesian coordinates in angstrom}
\end{array}$\\

\noindent Next the card contains DIMWANN lines giving the trial centers for the WFs.
Depending on the $<$TYPE$>$ flag at the beginning of the line,
formats are different.\\

\noindent $<$TYPE$>$ may assume the values: "atomic" , "1gauss", "2gauss" \\
\begin{displaymath}
\begin{array}{lllllll}
\textrm{IF ( TYPE == "atomic" )}  & \rightarrow
& \textrm{atomic}
&\quad \textrm{iatomic}
&\quad \textrm{l} \quad  \textrm{m}
&\quad \textrm{ }
&\quad \textrm{[weight]}\\
\textrm{IF ( TYPE == "1gauss" )}  & \rightarrow
& \textrm{1gauss}
&\quad \textrm{x} \quad  \textrm{y} \quad  \textrm{z}
&\quad \textrm{l} \quad  \textrm{m}
&\quad \textrm{rloc}
&\quad \textrm{[weight]} \\
\textrm{IF ( TYPE == "2gauss" )}  & \rightarrow
& \textrm{2gauss}
&\quad \textrm{x} \quad  \textrm{y} \quad  \textrm{z}
&\quad \textrm{xx} \quad  \textrm{yy} \quad  \textrm{zz}
&\quad \textrm{rloc}
&\quad \textrm{[weight]}
\end{array}
\end{displaymath}\bigskip

\noindent {\bf TYPE == "1gauss}"\\
\noindent The trial center is given by a single gaussian set at a given position with a given
angular momentum. Standard positions are usually atomic sites or bond midpoints.\\

\newdimen\descindent \descindent = 8pc
{\noindent \leftskip = \descindent \parskip = .5\baselineskip
\llap{\hbox to \descindent{\bf x, y, z\hfil}}%
REAL \\ define the position of the trial function. Units maybe specified aside
            the tag WANNIER\_CENTERS: see above for more details. \par

\noindent\llap{\hbox to \descindent{\bf l, m\hfil}}%
INTEGER \\ are the angular momentum quantum numbers for the spherical harmonics
giving the angular part of the trial WF. l can be set equal to 0, 1, or 2,
(and m values are then as usual) for standard spherical harmonics or l == -1
indicate the sp3 geometry. Here spherical harmonics are the real ones:\\
$\begin{array}{lllll}
\textrm{l == -1:}   &\textrm{m = -4}  &\rightarrow   & \textrm{ 1,1,-1}              &\textrm{dir}\\
\textrm{}           &\textrm{m = -3}  &\rightarrow   & \textrm{ 1,-1, 1}             &\textrm{dir}\\
\textrm{}           &\textrm{m = -2}  &\rightarrow   & \textrm{-1, 1,1}              &\textrm{dir}\\
\textrm{}           &\textrm{m = -1}  &\rightarrow   & \textrm{-1,-1,-1}             &\textrm{dir}\\
\textrm{}           &\textrm{m =  1}  &\rightarrow   & \textrm{ 1,1, 1}              &\textrm{dir}\\
\textrm{}           &\textrm{m =  2}  &\rightarrow   & \textrm{ 1,-1,-1}             &\textrm{dir}\\
\textrm{}           &\textrm{m =  3}  &\rightarrow   & \textrm{-1,1,-1}              &\textrm{dir}\\
\textrm{}           &\textrm{m =  4}  &\rightarrow   & \textrm{-1,-1, 1}             &\textrm{dir}\\
\textrm{l == 0:}    &\textrm{m =  0}  &\rightarrow   & \textrm{spherical}            &\textrm{}\\
\textrm{l == 1:}    &\textrm{m = -1}  &\rightarrow   & \textrm{x}                    &\textrm{}\\
\textrm{}           &\textrm{m =  0}  &\rightarrow   & \textrm{z}                    &\textrm{}\\
\textrm{}           &\textrm{m =  1}  &\rightarrow   & \textrm{y}                    &\textrm{}\\
\textrm{l == 2:}    &\textrm{m = -2}  &\rightarrow   & \textrm{x}^2 - \textrm{ y}^2  &\textrm{}\\
\textrm{}           &\textrm{m = -1}  &\rightarrow   & \textrm{xz}                   &\textrm{}\\
\textrm{}           &\textrm{m =  0}  &\rightarrow   & \textrm{3z}^2 - \textrm{ r}^2 &\textrm{}\\
\textrm{}           &\textrm{m =  1}  &\rightarrow   & \textrm{yz}                   &\textrm{}\\
\textrm{}           &\textrm{m =  2}  &\rightarrow   & \textrm{xy}                   &\textrm{}\\
\end{array}$ \par

\noindent\llap{\hbox to \descindent{\bf rloc\hfil}}%
REAL \\ specifies the spread of the gaussian used for the radial part of the
trial WF. Units are bohr for both "bohr" and "crystal" and angstrom for "angstrom" specifier.\par

\noindent\llap{\hbox to \descindent{\bf weight\hfil}}%
REAL \\ this value is required when conditioned minimization is performed. In case,
            it should be in the interval [0, 1] ans weights the relative importance of
            each center in the penalty functional. weight = 0 is used to switch off the
            constrain for a given center\par
}
\bigskip

\noindent {\bf TYPE == "2gauss}"\\
\noindent The trial function is given as the difference between gaussians with s-symmetry placed
at positions selected by the user. This is useful to mimic a antibonding state.\\

\newdimen\descindent \descindent = 8pc
{\noindent \leftskip = \descindent \parskip = .5\baselineskip
\llap{\hbox to \descindent{\bf x, y, z\hfil}}%
REAL \\ as before for TYPE == "1gauss" \par

\noindent\llap{\hbox to \descindent{\bf xx, yy, zz\hfil}}%
REAL \\ as before for x,y,z for the units, specify the center of a second gaussian
            used to build up the trial WF. This second case could be useful to describe
            anti-bonding WF.\par

\noindent\llap{\hbox to \descindent{\bf rloc\hfil}}%
REAL \\  as before for TYPE == "1gauss" \par

\noindent\llap{\hbox to \descindent{\bf weight\hfil}}%
REAL \\  as before for TYPE == "1gauss" \par
}
\bigskip

\noindent {\bf TYPE == "atomic}"\\
\noindent Atomic (pseudo)-orbitals from pseudopotential files are used as trial functions. They are
specified by the atomic index and the required angular momentum quantum numbers.\\

\newdimen\descindent \descindent = 8pc
{\noindent \leftskip = \descindent \parskip = .5\baselineskip
\llap{\hbox to \descindent{\bf iatom\hfil}}%
INTEGER \\ the index of the chosen atom, the same of Pwscf calculation \par

\noindent\llap{\hbox to \descindent{\bf l, m\hfil}}%
REAL \\ as before for TYPE == "1gauss"\par

\noindent\llap{\hbox to \descindent{\bf weight\hfil}}%
REAL \\  as before for TYPE == "1gauss" \par
}

\subsection{Input for electronic transport calculations}
\noindent {\bf Step 3 : conductor.x}\\
\noindent Both bulk and two-terminal calculations use similar input. Labels follows the scheme of Section \ref{section:run}.\\

\noindent Namelists allowed : \&INPUT\_CONDUCTOR\\
\noindent Cards allowed (tagged): $<$HAMILTONIAN\_DATA$>$, $<$subcard$>$ (xlm format)
\noindent Input file layout
\begin{description}
  \item \&INPUT\_CONDUCTOR
  \item ...
  \item $/$
  \item $<$HAMILTONIAN\_DATA$>$
  \item $\textrm{}\quad<$subcard$>$
  \item $\textrm{}\qquad$\&MATRIX\_DATA
  \item $\textrm{}\qquad$...
  \item $\textrm{}\qquad /$
  \item $\textrm{}\quad</$subcard$>$
  \item $</$HAMILTONIAN\_DATA$>$
\end{description}

\begin{centering}
\rule{2.2in}{0.01in} NAMELIST \&INPUT\_CONDUCTOR \rule{2.2in}{0.01in}
\end{centering}\\

\newdimen\descindent \descindent = 8pc
{\noindent \leftskip = \descindent \parskip = .5\baselineskip
\llap{\hbox to \descindent{\bf dimA\hfil}}%
INTEGER \\ number of sites in the lead A
\\ DEFAULT = 0 \par

\noindent\llap{\hbox to \descindent{\bf dimB\hfil}}%
INTEGER \\ number of sites in the lead B \\ DEFAULT = 0 \par

\noindent\llap{\hbox to \descindent{\bf dimC\hfil}}%
INTEGER \\ number of sites in the conductor C \\ DEFAULT = 0 \par

\noindent\llap{\hbox to \descindent{\bf calculation\_type\hfil}}%
STRING \\ ( "conductor" $\mid$ "bulk" )
            determines which kind of calculation should be performed:\\
            "conductor":  ordinary transport calculation for a leads$\mid$conductor$\mid$lead
                          interface\\
            "bulk": transport in a bulk system\\
            DEFAULT: "conductor" \par

\noindent\llap{\hbox to \descindent{\bf loverlap\hfil}}%
LOGICAL \\ If .TRUE. reads the overlap matrices from file, otherwise basis orthonormality
            is assumed (which is by definition the case of Wannier functions)\\
            DEFAULT : .FALSE. \par

\noindent\llap{\hbox to \descindent{\bf ne\hfil}}%
INTEGER \\ dimension of the energy grid for transmittance and spectral function
            calculation \\ DEFAULT = 1000 \par

\noindent\llap{\hbox to \descindent{\bf niterx\hfil}}%
INTEGER \\ maximum number of iterations in the calculation of transfer matrices \\ DEFAULT = 200 \par

\noindent\llap{\hbox to \descindent{\bf emin\hfil}}%
REAL \\ lower limit [eV] of the energy grid dimensioned by NE \\ DEFAULT = -10.0 \par

\noindent\llap{\hbox to \descindent{\bf emax\hfil}}%
REAL \\ upper limit [eV] of the energy grid dimensioned by NE \\ DEFAULT = +10.0 \par

\noindent\llap{\hbox to \descindent{\bf bias\hfil}}%
REAL \\ bias voltage [eV] across the conductor region. \\ DEFAULT = 0.0 \par

\noindent\llap{\hbox to \descindent{\bf delta\hfil}}%
REAL \\  small imaginary part used to get off the real axix in the calculation
            of Green's functions \\ DEFAULT = 1.0E-5 \par
}\bigskip

\begin{centering}
\rule{2.0in}{0.01in} CARD $<$HAMILTONIAN\_DATA$>$ \rule{2.0in}{0.01in}
\end{centering}\\

\noindent The tagged $<$HAMILTONIAN\_DATA$>$ is a mandatory card that specifies
the hamiltonian matrices to be used in the transport calculation. It includes necessary
``$<$subcards$>$'' (xlm format) to be used the order shown below. The name and the number of the subcards
depend on the ``calculation\_type'' flag:

\begin{displaymath}
\begin{array}{lll}
\textrm{IF ( calculation\_type = ``bulk'' )}       & \rightarrow & \textrm{two subcards are necessary}\\
\textrm{}                                          & \textrm{}   & <\textbf{H00\_C}>,<\textbf{HCI\_CB}>\\
\textrm{}                                          & \textrm{}   & \textbf{}\\
\textrm{IF ( calculation\_type = ``conductor'' )}  & \rightarrow & \textrm{six subcards are necessary}\\
\textrm{}                                          & \textrm{}   & <\textbf{H00\_C}>,<\textbf{HCI\_CB}>,<\textbf{HCI\_AC}>,\\
\textrm{}                                          & \textrm{}   & <\textbf{H00\_A}>,<\textbf{H01\_A}>,\\
\textrm{}                                          & \textrm{}   & <\textbf{H00\_B}>, <\textbf{H01\_B}>\\
\end{array}
\end{displaymath}

\noindent Names of the subcards refers to Figure \ref{acb:ham} of Sec. \ref{subsection:transport}. \\
\noindent Each subcard contains the namelist ``\&MATRIX\_DATA'' and the following variables:

\newdimen\descindent \descindent = 8pc
{\noindent \leftskip = \descindent \parskip = .5\baselineskip
\llap{\hbox to \descindent{\bf filename\hfil}}%
STRING ``prefix\_postfix.ham''\\
name of the file .ham, produced by wannier.x that includes the hamiltonian matrices\\
              DEFAULT = mandatory \par

\noindent\llap{\hbox to \descindent{\bf rows\hfil}}%
INTEGER \\ It is a string of format e.g "1-6" (analogous to the fmt used
              to specify pages to very common print tools).\\
It defines the number of rows to be extracted from the original
hamiltonian matrix, specified by ``filename''. The number of selected columns MUST be coherent case by case with the
dimension of the hamiltonian matrices defined by dimA, dimB, dimC (see Fig. \ref{acb:ham}
for details).\\
              DEFAULT = mandatory \par

\noindent\llap{\hbox to \descindent{\bf cols\hfil}}%
INTEGER \\ It is a string of format e.g "1-6" (analogous to the fmt used
              to specify pages to very common print tools).\\
It defines the number of columns to be extracted from the original
hamiltonian matrix, specified by ``filename''. The number of selected columns MUST be coherent case by case with the
dimension of the hamiltonian matrices defined by dimA, dimB, dimC (see Fig. \ref{acb:ham}
for details).\\
              DEFAULT = mandatory \par

\noindent\llap{\hbox to \descindent{\bf direction\hfil}}%
INTEGER \\ It defines the real space cell, over which we calculate the real space hamiltonian matrices.
The spatial direction  of the hopping matrices defines the direction of electronic transport\\
Allowed value are:\\

$\begin{array}{lll}
\textbf{R = 0}   & \rightarrow &  \textrm{On site H00 hamiltonian} \\
\textrm{}       & \textrm{}   &  \textrm{H00}_{mn}=\langle w_{m{\bf 0}} |\widehat{H} | w_{n{\bf 0}}\rangle\\
\textrm{}       & \textrm{}   &  \textrm{i.e. the hamiltonian matrix corresponding to WFs localized}\\
\textrm{}       & \textrm{}   &  \textrm{in the origin cell }\quad (\textbf{R = 0 0 0})\\
\textbf{R = 1}   & \rightarrow &  \textrm{Hopping H01 hamiltonian} \\
\textrm{}       & \textrm{}   &  \textrm{H01}_{mn}=\langle w_{m{\bf 0}} |\widehat{H} | w_{n{\bf 1}}\rangle\\
\textrm{}       & \textrm{}   &  \textrm{i.e. the hamiltonian matrix among to WFs localized}\\
\textrm{}       & \textrm{}   &  \textrm{in the origin cell and in the first neighbor cell along}\\
\textrm{}       & \textrm{}   &  \textrm{x direction }\quad (\textbf{R = 1 0 0})\\
\textbf{R = 2}   & \rightarrow &  \textrm{Hopping H01 hamiltonian} \\
\textrm{}       & \textrm{}   &  \textrm{H01}_{mn}=\langle w_{m{\bf 0}} |\widehat{H} | w_{n{\bf 1}}\rangle\\
\textrm{}       & \textrm{}   &  \textrm{i.e. the hamiltonian matrix among to WFs localized}\\
\textrm{}       & \textrm{}   &  \textrm{in the origin cell and in the first neighbor cell along}\\
\textrm{}       & \textrm{}   &  \textrm{y direction }\quad (\textbf{R = 0 1 0})\\
\textbf{R = 3}   & \rightarrow &  \textrm{Hopping H01 hamiltonian} \\
\textrm{}       & \textrm{}   &  \textrm{H01}_{mn}=\langle w_{m{\bf 0}} |\widehat{H} | w_{n{\bf 1}}\rangle\\
\textrm{}       & \textrm{}   &  \textrm{i.e. the hamiltonian matrix among to WFs localized}\\
\textrm{}       & \textrm{}   &  \textrm{in the origin cell and in the first neighbor cell along}\\
\textrm{}       & \textrm{}   &  \textrm{z direction }\quad (\textbf{R = 0 0 1})
\end{array}$ \par
}

\subsection{Input for post-processing calculations}
\subsubsection{bands.x}
\noindent Namelists allowed : \&INPUT\\
\noindent Input file layout
\begin{description}
  \item \&INPUT
  \item ...
  \item $/$
  \item kpt\_label\_1
  \item ...
  \item kpt\_label\_N
\end{description}

\begin{centering}
\rule{2.2in}{0.01in} NAMELIST \&INPUT \rule{2.2in}{0.01in}
\end{centering}\\

\newdimen\descindent \descindent = 8pc
{\noindent \leftskip = \descindent \parskip = .5\baselineskip
\llap{\hbox to \descindent{\bf prefix\hfil}}%
STRING \\ the first part of the name of all the file written by the code
              should be equal to the value given in the main calculations.\\
              DEFAULT = mandatory \par

\noindent\llap{\hbox to \descindent{\bf postfix\hfil}}%
STRING \\ the tail of the names of the above mentioned files (useful e.g. to
              distinguish among different calculations having a common part).
              should be equal to the value given in the main calculations.\\
              DEFAULT = "" \par

\noindent\llap{\hbox to \descindent{\bf work\_dir\hfil}}%
STRING \\ the scratch directory where the massive data file will be written\\
              DEFAULT = "$./$" \par

\noindent\llap{\hbox to \descindent{\bf verbosity\hfil}}%
STRING \\ ( "low" $\mid$ "medium" $\mid$ "high" )\\
              the level of detail of the code output\\
              DEFAULT =  "medium" \par

\noindent\llap{\hbox to \descindent{\bf nkpts\_in\hfil}}%
INTEGER \\ number of edge kpts defining the directions on which bands will
              be calculated\\ DEFAULT: mandatory \par

\noindent\llap{\hbox to \descindent{\bf nkpts\_max\hfil}}%
INTEGER \\ maximum number of interpolated kpoints\\ DEFAULT = 100 \par

\noindent\llap{\hbox to \descindent{\bf spin\_component\hfil}}%
STRING \\  ( "up" $\mid$ "down" $\mid$ "none" )\\
              define whether the calculation is spin polarized and if the case
              which spin component is to be treated.\\ DEFAULT = 100 \par
}\bigskip

\noindent After the INPUT namelist for each of the NSPTS kpts two lines with the following
format must be provided:\\

\noindent $\qquad \textrm{kpt\_label} \qquad\qquad   \textrm{kx} \quad  \textrm{ky} \quad \textrm{kz}$

\newdimen\descindent \descindent = 8pc
{\noindent \leftskip = \descindent \parskip = .5\baselineskip
\llap{\hbox to \descindent{\bf kpt\_label\hfil}}%
CHARACTER$(*)$ \\ it is a string with the name of the kpoint \par

\noindent\llap{\hbox to \descindent{\bf kx, ky, kz\hfil}}%
REAL \\ component of the kpt vector in units of crystal reciprocal
                 lattice vector \\ i.e. k = kx * b1 + ky * b2 + kz * b3 \par
}

\subsubsection{plot.x}
\noindent Namelists allowed : \&INPUT\\
\noindent Input file layout
\begin{description}
  \item \&INPUT
  \item ...
  \item $/$
\end{description}

\begin{centering}
\rule{2.2in}{0.01in} NAMELIST \&INPUT \rule{2.2in}{0.01in}
\end{centering}\\

\newdimen\descindent \descindent = 8pc
{\noindent \leftskip = \descindent \parskip = .5\baselineskip
\llap{\hbox to \descindent{\bf prefix\hfil}}%
STRING \\ the first part of the name of all the file written by the code
              should be equal to the value given in the main calculations.\\
              DEFAULT = mandatory \par

\noindent\llap{\hbox to \descindent{\bf postfix\hfil}}%
STRING \\ the tail of the names of the above mentioned files (useful e.g. to
              distinguish among different calculations having a common part).
              should be equal to the value given in the main calculations.\\
              DEFAULT = "" \par

\noindent\llap{\hbox to \descindent{\bf work\_dir\hfil}}%
STRING \\ the scratch directory where the massive data file will be written\\
              DEFAULT = "$./$" \par

\noindent\llap{\hbox to \descindent{\bf wann\hfil}}%
STRING \\ specifies the indexes of the Wannier functions to be plotted.
              It is a string of format e.g "1-3,5,7-9" (analogous to the fmt used
              to specify pages to very common print tools)\\
              DEFAULT = mandatory \par

\noindent\llap{\hbox to \descindent{\bf data\_type\hfil}}%
STRING \\ ("modulus" $\mid$ "real" $\mid$ "imaginary")\\
              specifies the type of data plotted:\\
                "modulus":    plot the real space square modulus of the WFs.
                "real":       plot the real part (in real space) of the WFs.
                "imaginary":  plot the imaginary part (in real space) of the WFs
                              this choice shouldbe intended as a check because WFs
                              are expected to be more or less "real".\\
              DEFAULT = "modulus" \par

\noindent\llap{\hbox to \descindent{\bf output\_fmt\hfil}}%
STRING \\ ( "plt" $\mid$ "txt" $\mid$ "cube" )\\
              Define the format of the output file. PLT is binary and smaller than
              CUBE and TXT. While CUBE deals also with non-orthorombic lattices, TXT
              is suitable to be converted to further format.\\
              DEFAULT = "plt" \par

\noindent\llap{\hbox to \descindent{\bf r1min, r1max\hfil}}%
REAL \\ the starting and ending points of the plotting cell along a1 dir,
              in units of a1 lattice vector (crystal coord).\\
              DEFAULT = -0.5, 0.5 \par

\noindent\llap{\hbox to \descindent{\bf r2min, r2max\hfil}}%
REAL \\ as before but for a2 direction.\\
              DEFAULT = -0.5, 0.5 \par

\noindent\llap{\hbox to \descindent{\bf r3min, r3max\hfil}}%
REAL \\ as before but for a3 direction.\\
              DEFAULT = -0.5, 0.5 \par

\noindent\llap{\hbox to \descindent{\bf assume\_ncpp\hfil}}%
LOGICAL \\ if using DFT pseudoptentials not readable in \WANT\ set this value to
              .TRUE. in order to avoid PP reading.\\
              DEFAULT = ".FALSE." \par

\noindent\llap{\hbox to \descindent{\bf locate\_wf\hfil}}%
LOGICAL \\ if .TRUE. move the WFs in a unit cell centered around the midpoint of the
              plotting cell. Useful to plot purposes.\\
              DEFAULT = ".TRUE." \par

\noindent\llap{\hbox to \descindent{\bf spin\_component\hfil}}%
STRING \\ ( "up" $\mid$ "down" $\mid$ "none" )\\
              define whether the calculation is spin polarized and if the case
              which spin component is to be treated.\\
              DEFAULT = "none" \par
}
