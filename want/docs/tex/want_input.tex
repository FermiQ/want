%%%%%%%%%%%%%%%%%%%%%%%%%%%%%%%%%%%%%%%%%%%%%%%%%%%%%%%%%%
%  Copyright (C) 2005 WanT group                         %
%  This file is distributed under the terms of the       %
%  GNU General Public License.                           %
%  See the file `License'  in the root directory of      %
%  the present distribution,                             %
%  or http://www.gnu.org/copyleft/gpl.txt                %
%%%%%%%%%%%%%%%%%%%%%%%%%%%%%%%%%%%%%%%%%%%%%%%%%%%%%%%%%%

\thispagestyle{empty}
\section{How to setup input files}\label{sec:input}

\noindent According to the methodological scheme of Sec.~\ref{section:run}, 
it is necessary to use separate input files at
the different step of the \WANT\ procedure.\\

\noindent Input files are organized using several {\tt NAMELIST}s,
followed by other fields with more massive data {\tt CARDS}. Namelists are
begin with the flag {\tt \&NAMELIST } and end with the
"$/$'' bar. The order of variables within a namelist is
arbitrary. Most variables have default values mandatory. 
If a variable is not explicitly defined in the input file, 
its default value is assumed. Other variables are mandatory and must be
always supplied.
In the following we report the list and the description
of the details of each required input file.

\subsection{Input for DFT-PW calculations}
\noindent {\bf Step 1. i-ii:} {\tt pw.x}
\\
\noindent \WANT\ is currently interfaced with \PWSCF code. For the
description of the input for steps 1-2 (Sec. \ref{section:run})
and for further details see the \PWSCF manual at \PWSCFURL .\\

\noindent {\bf Step 1. iii:}  {\tt pw\_export.x}
\\
\noindent Input file layout: \\

%
%
\begin{tabular}{c}
  {\tt \&INPUTPP } \\
    ... \\
  $/$
\end{tabular}
%
%
\\

%
%
\begin{centering}
\rule{2.5in}{0.01in} List of variables \rule{2.5in}{0.01in}
\end{centering}\\

\newdimen\descindent \descindent = 8pc
{\noindent \leftskip = \descindent \parskip = .5\baselineskip
\llap{\hbox to \descindent{\tt prefix\hfil}}%
{\sc string} \\ the first part of the name of all the files written by
the code. When using \PWSCF for the DFT calculation, {\tt prefix} should be the
same.\\ {\sc default} = {\tt mandatory} \par

\noindent\llap{\hbox to \descindent{\tt outdir\hfil}}%
{\sc string} \\ the scratch directory where the massive data-files will
be written.\\  {\sc default} = {\tt "./" }\par

\noindent\llap{\hbox to \descindent{\tt pseudo\_dir\hfil}}%
{\sc string} \\directory containing pseudopotential (PP) files.\\
{\sc default} = {\tt "./"} \par

\noindent\llap{\hbox to \descindent{\tt psfile(i)\hfil}}%
{\sc string} \\ files containing $i$-th pseudopotential, where $i=1, N_{\text{type}}$. 
PP numbering must follow the ordering defined in the input of {\tt pw.x} . \\
{\sc default} =  {\tt ""} \par
\par

\noindent\llap{\hbox to \descindent{\tt single\_file\hfil}}%
{\sc logical} \\ if {\tt .TRUE.} a single output file is produced, otherwise
  the output is a directory with few files.\\ 
{\sc default} = {\tt .FALSE. }\par

\noindent\llap{\hbox to \descindent{\tt ascii\hfil}}%
{\sc logical}  \\ if {\tt .TRUE.} output files are textual, otherwise
they are partly binary.\\ 
{\sc default} = {\tt .FALSE.} \par}\bigskip


%%%%%%%%%%%%%%%%%%%%%%%%%%%%%%%%%%%%%%%%%%%%%%%%%%%%%%%%%%%%
\subsection{Input for Wannier function calculations}
\noindent {\bf Step 2. a-b:} {\tt disentangle.x wannier.x}\\
\noindent Both codes for WF calculation use the same input file.\\

\noindent Input file layout: \\

%
%
\begin{tabular}{l}
{\tt \&CONTROL }\\
   ... \\
  / \\
  \\
{\tt \&SUBSPACE }\\
   ... \\
  $/$ \\
  \\
{\tt \&LOCALIZATION }\\
   ... \\
  $/$ \\
  \\
{\tt  WANNIER\_CENTERS } {\tt ( "crystal" $\mid$ "angstrom" $\mid$ "bohr" ) }\\
  $\langle${\tt type}$_1\rangle \qquad    \langle${\tt specific\_fmt}$\rangle$ \\
  ... \\
  $\langle${\tt type}$_N\rangle \qquad    \langle${\tt specific\_fmt}$\rangle$ 
\end{tabular}
%
%
\\
\\

\begin{centering}
\rule{2.2in}{0.01in} Namelist {\tt \&CONTROL } \rule{2.2in}{0.01in}
\end{centering}\\

\newdimen\descindent \descindent = 8pc
{\noindent \leftskip = \descindent \parskip = .5\baselineskip
\llap{\hbox to \descindent{\tt prefix\hfil}}%
{\sc string} \\ the first part of the name of all the files written by
the code.\\ {\sc default} = {\tt mandatory} \par

\noindent\llap{\hbox to \descindent{\tt postfix\hfil}}%
{\sc string} \\ the tail of the names of the above mentioned files (useful {\it e.g.} to
distinguish among different calculations having a common part).\\ 
{\sc default} = {\tt ""} \par

\noindent\llap{\hbox to \descindent{\tt work\_dir\hfil}}%
{\sc string} \\ the scratch directory where the massive data-files will be written.\\ 
{\sc default} = {\tt "$./$" }\par

\noindent\llap{\hbox to \descindent{\tt title\hfil}}%
{\sc string} \\ the title of the calculation.\\ 
{\sc default} = {\tt "Wannier Transport Calculation"} \par

\noindent\llap{\hbox to \descindent{\tt restart\_mode\hfil}}%
{\sc string} \\ {\tt ( "from\_scratch" $\mid$ "restart" ) }\\
define whether to restart a previous calculation;
at the moment the {\tt "restart"} choice implies to give the value
{\tt "from\_file"} to the input 
variables {\tt overlaps, projections,
start\_mode\_dis} and {\tt start\_mode\_wan}
(see below for thier meanings).\\ {\sc default} = {\tt "from\_scratch"} \par

\noindent\llap{\hbox to \descindent{\tt verbosity\hfil}}%
{\sc string} \\ {\tt ( "low" $\mid$ "medium" $\mid$ "high" ) }
\\the level of detail of the textual output files.\\ 
{\sc default} = {\tt "medium"} \par

\noindent\llap{\hbox to \descindent{\tt overlaps\hfil}}%
{\sc string} \\ {\tt ( "from\_scratch" $\mid$ "from\_file" ) }\\
determine how to get overlap integrals:\\
{\tt "from\_scratch"}:  overlaps are calculated from wfcs\\
{\tt "from\_file"}:     overlaps are read from a previous data file.
In this second case the dimensions of the problem should be the same as in the
original calculation.\\ 
{\sc default} = {\tt "from\_scratch"} \par

\noindent\llap{\hbox to \descindent{\tt projections\hfil}}%
{\sc string} \\ {\tt ( "from\_scratch" $\mid$ "from\_file" ) }\\
determine how to get projections integrals:
the meaning of the options is as for {\tt overlaps} variable.\\ 
{\sc default} = {\tt "from\_scratch"} \par

\noindent\llap{\hbox to \descindent{\tt assume\_ncpp\hfil}}%
{\sc logical} \\ if {\tt .TRUE.} avoids the reading of pseudopotential files
assuming that the DFT calculation has been performed using norm-conserving
pseudopotentials (no knowledge of them is required in the \WANT\ calculation 
in this case).\\ 
{\sc default} = {\tt .FALSE.} \par

\noindent\llap{\hbox to \descindent{\tt unitary\_thr\hfil}}%
{\sc real} \\ threshold for the check of matrix unitariery.\\ 
{\sc default} = {\tt 1.0d-6} \par
}\bigskip

\begin{centering}
\rule{2.2in}{0.01in} Namelist {\tt \&SUBSPACE} \rule{2.2in}{0.01in}
\end{centering}\\

\newdimen\descindent \descindent = 8pc
{\noindent \leftskip = \descindent \parskip = .5\baselineskip
\llap{\hbox to \descindent{\tt dimwann\hfil}}%
{\sc integer} \\ the number of Wannier functions, {\it i.e.} the dimension of the 
Wannier subspace.
\\ {\sc default} = {\tt mandatory} \par

\noindent\llap{\hbox to \descindent{\tt win\_min\hfil}}%
{\sc real} \\ the lower limit [eV] of the energy window containing the Bloch states
forming the Wannier subspace).\\ 
{\sc default} = {\tt $-\infty$} \par

\noindent\llap{\hbox to \descindent{\tt win\_max\hfil}}%
{\sc real} \\ the upper limit [eV] of the energy window described above.\\ 
{\sc default} = {\tt $+\infty$} \par

\noindent\llap{\hbox to \descindent{\tt froz\_min\hfil}}%
{\sc real} \\ the lower limit [eV] of the energy window containing {\it frozen} 
Bloch states: they will be taken as they are in the Wannier subspace and do not enter
the disentanglement procedure. \\
{\sc default} = {\tt $-\infty$} \par

\noindent\llap{\hbox to \descindent{\tt froz\_max\hfil}}%
{\sc real} \\ upper limit [eV] of the frozen window described above.\\ 
{\sc default} = {\tt $-\infty$} \par

\noindent\llap{\hbox to \descindent{\tt alpha\_dis\hfil}}%
{\sc real} \\ mixing parameter for the disentangle iterative procedure.\\ 
{\sc default} = {\tt 0.5} \par

\noindent\llap{\hbox to \descindent{\tt maxiter\_dis\hfil}}%
{\sc integer} \\  maximum number of iterations during the disentangle procedure.\\ 
{\sc default} = {\tt 1000} \par

\noindent\llap{\hbox to \descindent{\tt nprint\_dis\hfil}}%
{\sc integer} \\  every {\tt nprint\_dis} iterations in disentangle minimization write to 
stdout.\\ {\sc default} = {\tt 10} \par

\noindent\llap{\hbox to \descindent{\tt nsave\_dis\hfil}}%
{\sc integer} \\  every {\tt nsave\_dis} iterations save subspace data to disk.\\ 
{\sc default} = {\tt 10} \par

\noindent\llap{\hbox to \descindent{\tt use\_blimit\hfil}}%
{\sc logical} \\   if {\tt .TRUE.}, $\mathbf{b}$-vectors are set to zero 
when calculating overlap augmentations.
This essentially means we are doing a sort of thermodynamic limit
even if this is not consistent with the actual kpt grid. The {\tt .TRUE.} value
should be considered for debug purposes.\\ {\sc default} = {\tt .FALSE.} \par

\noindent\llap{\hbox to \descindent{\tt disentangle\_thr\hfil}}%
{\sc real} \\  threshold for convergence of the iterative disentangle procedure.\\ 
{\sc default} = {\tt 1.0d-8} \par

\noindent\llap{\hbox to \descindent{\tt subspace\_init\hfil}}%
{\sc string} \\  {\tt ( "randomized" $\mid$ "lower\_states" $\mid$ "upper\_states" $\mid$ \\
"center\_projections" $\mid$ "from\_file" ) }\\
Determine how the trial subspace is chosen\\
{\tt "randomized"}   : random starting point is chosen\\
{\tt "lower\_states"} : the lower {\tt dimwann} bands from DFT calculation are
                 used to define the subspace\\
{\tt "upper\_states"} : the upper {\tt dimwann} bands from DFT calculation are
                 used to define the subspace\\
{\tt "center\_projections"} : a subspace is extracted from the DFT bands
                 by means of a projections on the given Wannier trial-centers
                 (see the card {\tt WANNIER\_CENTERS})\\
{\tt "from\_file"} : subspace initialization is read from an existing data file;
                 this is the choice used during restart.\\
{\sc default} = {\tt "center\_projections"} \par

\noindent\llap{\hbox to \descindent{\tt spin\_component\hfil}}%
{\sc string} \\  {\tt ( "up" $\mid$ "down" $\mid$ "none")}\\
defines whether the calculation is spin polarized and if the case
which spin component is to be treated.\\ 
{\sc default} = {\tt "none"} \par
}\bigskip

\begin{centering}
\rule{2.2in}{0.01in} Namelist {\tt \&LOCALIZATION} \rule{2.2in}{0.01in}
\end{centering}\\

\newdimen\descindent \descindent = 8pc
{\noindent \leftskip = \descindent \parskip = .5\baselineskip
\llap{\hbox to \descindent{\tt wannier\_thr\hfil}}%
{\sc real} \\ threshold for convergence of the iterative Wannier minimization.
\\ {\sc default} = {\tt 1.0d-6} \par

\noindent\llap{\hbox to \descindent{\tt alpha0\_wan\hfil}}%
{\sc real} \\ mixing parameter during the steepest-descent minimization steps. \\
{\sc default} = {\tt 0.5} \par

\noindent\llap{\hbox to \descindent{\tt alpha1\_wan\hfil}}%
{\sc real} \\ mixing parameter during the conjugate-gradient minimization steps. \\ 
{\sc default} = {\tt 0.5} \par

\noindent\llap{\hbox to \descindent{\tt maxiter0\_wan\hfil}}%
{\sc integer} \\ maximum number of steps for the steepest descent 
minimization (first part).\\ {\sc default} = {\tt 500} \par

\noindent\llap{\hbox to \descindent{\tt maxiter1\_wan\hfil}}%
{\sc integer} \\ maximum number of steps for the conjugate-gradient minimization
(second part).\\ {\sc default} = {\tt 500 }\par

\noindent\llap{\hbox to \descindent{\tt nprint\_wan\hfil}}%
{\sc integer} \\ every {\tt nprint\_wan} iterations in wannier minimization write to stdout.\\ 
{\sc default} = {\tt 10 }\par

\noindent\llap{\hbox to \descindent{\tt nsave\_wan\hfil}}%
{\sc integer} \\ every {\tt nsave\_wan} iterations save subspace data to disk.\\ 
{\sc default} = {\tt 10 }\par

\noindent\llap{\hbox to \descindent{\tt ncg\hfil}}%
{\sc integer} \\ every {\tt ncg} iterations in the second minimization part, 
do a steepest-descent step.\\ {\sc default} = {\tt 3 }\par

\noindent\llap{\hbox to \descindent{\tt localization\_init\hfil}}%
{\sc $\quad$ string} \\  
{\tt ( "no\_guess" $\mid$ "randomized" $\mid$ "center\_projections" 
$\mid$ "from\_file" )}\\
 Determine how the Wannier localization is started\\
{\tt "no\_guess"} : disentangle states are used as starting point
                 without any further localization guess\\
{\tt "randomized"} : a random rotation is applied to the states found by
                 the disentangle procedure\\
{\tt "center\_projections"} : a subspace is extracted from the DFT bands
                 by means of a projections on the given Wannier trial-centers \\
{\tt "from\_file"} : subspace initialization is read from an existing data file;
                 this is the choice used during restart. \\
{\sc default} = {\tt "center\_projections"} \par

\noindent\llap{\hbox to \descindent{\tt ordering\_mode\hfil}}%
{\sc string} \\ {\tt ( "none" $\mid$ "spatial" $\mid$ "spread" $\mid$ "complete" ) } \\
specifies whether to order the computed Wannier functions and
              which ordering criterion adopt\\
{\tt "none"}:      no ordering is performed\\
{\tt "spatial"}:   ordering based on WF center positions\\
{\tt "spread"}:    ordering based on WF increasing spreads\\
{\tt "complete"}:  {\tt spatial} + {\tt spread}. \\
{\sc default} = {\tt "none" }\par

\noindent\llap{\hbox to \descindent{\tt a\_condmin\hfil}}%
{\sc real} \\ the amplitude of the conditioned minimization functional. If set to zero
              ordinary minimization is performed.\\ 
{\sc default} = {\tt 0.0} \par

\noindent\llap{\hbox to \descindent{\tt niter\_condmin\hfil}}%
{\sc integer} \\ the number of steps for which minimization is conditioned.\\
{\sc default} = $\begin{array}{ll}
           {\tt maxiter0\_wan + maxiter1\_wan}     & \textrm{(if {\tt a\_condmin}} \neq 0.0)\\
           {\tt 0}                                 & \textrm{(otherwise)}
           \end{array}$ \par

\noindent\llap{\hbox to \descindent{\tt dump\_condmin\hfil}}%
{\sc real} \\ the dumping factor for {\tt a\_condmin} during the conditioned minimization.
              If the dumping factor is specified, after {\tt niter\_condmin} 
              iterations {\tt a\_condmin}
              is dumped according to {\tt a\_condmin = a\_condmin $*$ dump\_condmin}
              at each iteration.\\
{\sc default} = {\tt 0.0} \par
}
\bigskip

\begin{centering}
\rule{2.0in}{0.01in} Card {\tt WANNIER\_CENTERS} \rule{2.0in}{0.01in}
\end{centering}\\

\noindent {\tt WANNIER\_CENTERS}
{\tt ( "crystal" $\mid$ "angstrom" $\mid$ "bohr" ) }\\

\noindent Aside the tag {\tt WANNIER\_CENTERS}, units for positions maybe specified:\\
%
%
\begin{tabular}{ll}
\texttt{"crystal"}  & : relative coordinates on the basis of $\mathbf{a}_1,\mathbf{a}_2,\mathbf{a}_3$
                                direct lattice vector (default)\\
\texttt{"bohr"}     & : cartesian coordinates in Bohr\\
\texttt{"angstrom"} & : cartesian coordinates in Angstrom
\end{tabular}
%
%
\\

\noindent Next the card contains {\tt dimwann} lines giving the trial centers for the WFs.
Depending on the $\langle${\tt type}$\rangle$ flag at the beginning of the line,
formats are different.\\

\noindent $\langle${\tt type}$\rangle$ may assume the values: {\tt "atomic" , "1gauss", "2gauss" }\\
%
%
\begin{displaymath}
\begin{array}{lllllll}
\texttt{if ( type == "atomic" )}  & \rightarrow
& \texttt{atomic}
&\quad \texttt{iatomic}
&\quad \texttt{l} \quad  \texttt{m}
&\quad \texttt{ }
&\quad \texttt{[weight]}\\
\texttt{if ( type == "1gauss" )}  & \rightarrow
& \texttt{1gauss}
&\quad \texttt{x} \quad  \texttt{y} \quad  \texttt{z}
&\quad \texttt{l} \quad  \texttt{m}
&\quad \texttt{rloc}
&\quad \texttt{[weight]} \\
\texttt{if ( type == "2gauss" )}  & \rightarrow
& \texttt{2gauss}
&\quad \texttt{x} \quad  \texttt{y} \quad  \texttt{z}
&\quad \texttt{xx} \quad  \texttt{yy} \quad  \texttt{zz}
&\quad \texttt{rloc}
&\quad \texttt{[weight]}
\end{array}
\end{displaymath}
%
%
\bigskip

\noindent {\tt type == "1gauss}"\\
\noindent The trial center is given by a single gaussian set at a given position with a given
angular momentum. Standard positions are usually atomic sites or bond midpoints.\\

\newdimen\descindent \descindent = 8pc
{\noindent \leftskip = \descindent \parskip = .5\baselineskip
\llap{\hbox to \descindent{\tt x, y, z\hfil}}%
{\sc real} \\ define the position of the trial function. Units maybe specified aside
            the tag {\tt WANNIER\_CENTERS}, see above for more details. \par

\noindent\llap{\hbox to \descindent{\tt l, m\hfil}}%
{\sc integer} \\ are the angular momentum quantum numbers for the spherical harmonics
giving the angular part of the trial center. {\tt l} can be set equal to 0, 1, or 2,
(and {\tt m} values are then as usual) for standard spherical harmonics or {\tt l} == -1
indicating the sp3 geometry. Here spherical harmonics are the real ones:\\

%
%
$\begin{array}{lllll}
\texttt{l == -1:}   &\texttt{m = -4}  &\rightarrow   & \texttt{ 1,1,-1}              &\texttt{dir}\\
\texttt{}           &\texttt{m = -3}  &\rightarrow   & \texttt{ 1,-1, 1}             &\texttt{dir}\\
\texttt{}           &\texttt{m = -2}  &\rightarrow   & \texttt{-1, 1,1}              &\texttt{dir}\\
\texttt{}           &\texttt{m = -1}  &\rightarrow   & \texttt{-1,-1,-1}             &\texttt{dir}\\
\texttt{}           &\texttt{m =  1}  &\rightarrow   & \texttt{ 1,1, 1}              &\texttt{dir}\\
\texttt{}           &\texttt{m =  2}  &\rightarrow   & \texttt{ 1,-1,-1}             &\texttt{dir}\\
\texttt{}           &\texttt{m =  3}  &\rightarrow   & \texttt{-1,1,-1}              &\texttt{dir}\\
\texttt{}           &\texttt{m =  4}  &\rightarrow   & \texttt{-1,-1, 1}             &\texttt{dir}\\
\texttt{l == 0:}    &\texttt{m =  0}  &\rightarrow   & \texttt{spherical}            &\texttt{}\\
\texttt{l == 1:}    &\texttt{m = -1}  &\rightarrow   & \texttt{x}                    &\texttt{}\\
\texttt{}           &\texttt{m =  0}  &\rightarrow   & \texttt{z}                    &\texttt{}\\
\texttt{}           &\texttt{m =  1}  &\rightarrow   & \texttt{y}                    &\texttt{}\\
\texttt{l == 2:}    &\texttt{m = -2}  &\rightarrow   & \texttt{x}^2 - \texttt{ y}^2  &\texttt{}\\
\texttt{}           &\texttt{m = -1}  &\rightarrow   & \texttt{xz}                   &\texttt{}\\
\texttt{}           &\texttt{m =  0}  &\rightarrow   & \texttt{3z}^2 - \texttt{ r}^2 &\texttt{}\\
\texttt{}           &\texttt{m =  1}  &\rightarrow   & \texttt{yz}                   &\texttt{}\\
\texttt{}           &\texttt{m =  2}  &\rightarrow   & \texttt{xy}                   &\texttt{}\\
\end{array}$ \par

\noindent\llap{\hbox to \descindent{\tt rloc\hfil}}%
{\sc real} \\ specifies the spread of the gaussian used for the radial part of the
trial WF. Units are Bohr for both {\tt"bohr"} and {\tt"crystal"} and Angstrom for {\tt "angstrom"} 
specifier.\par

\noindent\llap{\hbox to \descindent{\tt weight\hfil}}%
{\sc real} \\ this value is required when conditioned minimization is performed. In case,
            it should be set in the interval [0, 1]. It weights the relative importance of
            each center in the penalty functional. Weight = 0 is used to switch off the
            constrain for a given center.\par
}
\bigskip

\noindent {\tt type == "2gauss}"\\
\noindent The trial function is given as the difference between s-symmetry gaussians placed
at the positions selected by the user. This is useful to mimic some antibonding state.\\

\newdimen\descindent \descindent = 8pc
{\noindent \leftskip = \descindent \parskip = .5\baselineskip
\llap{\hbox to \descindent{\tt x, y, z\hfil}}%
{\sc real} \\ as before for {\tt type == "1gauss"}. \par

\noindent\llap{\hbox to \descindent{\tt xx, yy, zz\hfil}}%
{\sc real} \\ as before for {\tt x,y,z} but specify the center of the second gaussian
            used to set up the trial center.  \par

\noindent\llap{\hbox to \descindent{\tt rloc\hfil}}%
{\sc real} \\  as before for {\tt type == "1gauss"}. \par

\noindent\llap{\hbox to \descindent{\tt weight\hfil}}%
{\sc real} \\  as before for {\tt type == "1gauss"}. \par
}
\bigskip

\noindent {\tt type == "atomic}"\\
\noindent Atomic (pseudo)-orbitals from pseudopotential data-files are used as trial functions. 
They are specified by the atomic index and the required angular momentum quantum-numbers.\\

\newdimen\descindent \descindent = 8pc
{\noindent \leftskip = \descindent \parskip = .5\baselineskip
\llap{\hbox to \descindent{\tt iatom\hfil}}%
{\sc integer} \\ the index of the chosen atom (ordering is directly taken from DFT data). \par

\noindent\llap{\hbox to \descindent{\tt l, m\hfil}}%
{\sc real} \\ as before for {\tt type == "1gauss"}.\par

\noindent\llap{\hbox to \descindent{\tt weight\hfil}}%
{\sc real} \\ as before for {\tt type == "1gauss"}.\par
}


%%%%%%%%%%%%%%%%%%%%%%%%%%%%%%%%%%%%%%%%%%%%%%%%%%%%%%%%%%%%%%%%%%%%%%%
\subsection{Input for electronic transport calculations}
\noindent {\bf Step 3 : conductor.x}\\
\noindent Both bulk and two-terminal calculations use similar input. 
Labels follow the scheme of Sec.~\ref{section:run}.\\

\noindent Input file layout: \\

%
%
\begin{tabular}{l}
{\tt \&INPUT\_CONDUCTOR }\\
   ... \\
  / \\
  \\
{\tt <HAMILTONIAN\_DATA>} \\
  $\quad$ {\tt <ham$_1$   attr="" />} \\
  $\quad$ ...  \\
  $\quad$ {\tt <ham$_N$   attr="" />} \\
{\tt </HAMILTONIAN\_DATA>}
%
\end{tabular}
%
%
\\

\begin{centering}
\rule{2.2in}{0.01in} Namelist {\tt \&INPUT\_CONDUCTOR} \rule{2.2in}{0.01in}
\end{centering}\\

\newdimen\descindent \descindent = 8pc
{\noindent \leftskip = \descindent \parskip = .5\baselineskip
\llap{\hbox to \descindent{\tt dimL\hfil}}%
{\sc integer} \\ number of sites in the L-lead.
\\ {\sc default} = {\tt 0} \par

\noindent\llap{\hbox to \descindent{\tt dimR\hfil}}%
{\sc integer} \\ number of sites in the R-lead. \\ 
{\sc default} = {\tt 0} \par

\noindent\llap{\hbox to \descindent{\tt dimC\hfil}}%
{\sc integer} \\ number of sites in the conductor region. \\ 
{\sc default} = {\tt mandatory} \par

\noindent\llap{\hbox to \descindent{\tt calculation\_type\hfil}}%
{\sc string} \\ {\tt ( "conductor" $\mid$ "bulk" ) }
            determines the kind of calculation to be performed:\\
            {\tt "conductor"}:  ordinary transport calculation for a 
                    leads$\mid$conductor$\mid$lead junction\\
            {\tt "bulk"}: transport in a bulk system.\\
            {\sc default}: {\tt "conductor"} \par

\noindent\llap{\hbox to \descindent{\tt transport\_dir\hfil}}%
{\sc integer} \\ transport direction according to crystal axis indexing. \\
{\sc default} = {\tt mandatory} \par

\noindent\llap{\hbox to \descindent{\tt conduct\_formula\hfil}}%
{\sc string} \\ {\tt ( "landauer" $\mid$ "generalized" ) }
            {\tt "landauer"}:  transport is computed using the standard Landauer formula \\
            {\tt "generalized"}: a generalized Landauer formula accounting for a 
                 specific correlation correction is used. This case is experiemntal, see
                 Ref.~\cite{ferr+05prl, ferr+05prb} for more details. \\
            {\sc default}: {\tt "landauer"} \par

\noindent\llap{\hbox to \descindent{\tt ne\hfil}}%
{\sc integer} \\ dimension of the energy grid for transmittance and spectral function
            calculation.\\ 
{\sc default} = {\tt 1000}\par

\noindent\llap{\hbox to \descindent{\tt emin\hfil}}%
{\sc real} \\ lower limit [eV] of the energy grid. \\
{\sc default} = {\tt -10.0} \par

\noindent\llap{\hbox to \descindent{\tt emax\hfil}}%
{\sc real} \\ upper limit [eV] of the energy grid. \\
{\sc default} = {\tt +10.0} \par

\noindent\llap{\hbox to \descindent{\tt delta\hfil}}%
{\sc real} \\ small imaginary part used to get off the real axis in the calculation
            of Green's functions. \\ 
{\sc default} = {\tt 1.0e-5} \par

\noindent\llap{\hbox to \descindent{\tt nprint\hfil}}%
{\sc integer} \\ every {\tt nprint} energy step write to stdout .\\
{\sc default} = {\tt 20} \par

\noindent\llap{\hbox to \descindent{\tt niterx\hfil}}%
{\sc integer} \\ maximum number of iterations in the calculation of transfer 
                 matrices. \\ 
{\sc default} = { \tt 200} \par

\noindent\llap{\hbox to \descindent{\tt use\_overlap\hfil}}%
{\sc logical} \\ If {\tt .TRUE.} reads the overlap matrices from file, 
otherwise basis orthonormality
            is assumed (which is by definition the case of Wannier functions). \\
            {\sc default} : {\tt .FALSE.} \par

\noindent\llap{\hbox to \descindent{\tt use\_correlation\hfil}}%
{\sc logical} \\ If {\tt .TRUE.} correlation corrections are read from file and included
            in the calculation. See also the {\tt datafile\_sgm} variable. \\
            {\sc default} : {\tt .FALSE.} \par

\noindent\llap{\hbox to \descindent{\tt datafile\_C \hfil}}%
{\sc string} \\ Name of the file containing the Wannier Hamiltonian blocks for the
            conductor region. \\
            {\sc default}: {\tt "mandatory"} \par

\noindent\llap{\hbox to \descindent{\tt datafile\_L \hfil}}%
{\sc string} \\ Name of the file containing the Wannier Hamiltonian blocks for the
            L-lead. It is nor required for {\tt bulk} calculations. \\
            {\sc default}: {\tt "mandatory"} if not {\tt calculation\_type == "bulk"} \par

\noindent\llap{\hbox to \descindent{\tt datafile\_R \hfil}}%
{\sc string} \\ as for {\tt datafile\_L} but for R-lead. \\
            {\sc default}: {\tt "mandatory"} if not {\tt calculation\_type == "bulk"} \par

\noindent\llap{\hbox to \descindent{\tt datafile\_sgm \hfil}}%
{\sc string} \\ Name of the file containing the correlation self-energy. It is required only 
            when correlation is included in the calculation. \\
            {\sc default}: {\tt "mandatory"} if {\tt use\_correlation == .TRUE.} \par
}
\bigskip

\begin{centering}
\rule{2.0in}{0.01in} Card {\tt <HAMILTONIAN\_DATA>} \rule{2.0in}{0.01in}
\end{centering}\\

\noindent The card {\tt <HAMILTONIAN\_DATA>} is mandatory and specifies
the details about hamiltonian blocks to be used in transport calculation. 
It includes a variable number of subtags (XML format)
to be used the order shown below. The name and the number of these subcards
depend on the {\tt calculation\_type} variable:

%
%
\begin{displaymath}
\begin{array}{lll}
\texttt{if ( calculation\_type = "bulk" )}         & \rightarrow & 
                                                     \textrm{two subcards are needed}\\
\texttt{}                                          & \texttt{}   & 
                                                     \texttt{<H00\_C>},\texttt{<H\_CR>}\\
\texttt{}                                          & \texttt{}   & \textbf{}\\
\texttt{if ( calculation\_type = "conductor" )}    & \rightarrow & 
                                                     \textrm{seven subcards are needed}\\

\texttt{}                                          & \texttt{}   & 
                                                     \texttt{<H00\_C>},
                                                     \texttt{<H\_CR>},\texttt{<H\_LC>},\\
\texttt{}                                          & \texttt{}   & 
                                                     \texttt{<H00\_L>},\texttt{<H01\_L>},\\
\texttt{}                                          & \texttt{}   & 
                                                     \texttt{<H00\_R>}, \texttt{<H01\_R>}\\
\end{array}
\end{displaymath}
%
%

\noindent Names of the subcards refer to Fig.~\ref{fig:matrix_naming} in 
Sec.~\ref{subsec:tran}. 
Each subcard (tag) may contain a number of attribute, according to the 
format (XML compliant):
%
%
\begin{description}
\item
   {\tt <H$_{\text{whatever}}$  cols=""  rows=""  />}
\end{description}
%
%


\newdimen\descindent \descindent = 8pc
{\noindent \leftskip = \descindent \parskip = .5\baselineskip
\llap{\hbox to \descindent{\tt cols (rows) \hfil}}%
{\sc string} \\
the string describing which index should be considered to define the 
columns (rows) of the specific {\tt H} submatrix.
The format in the string is quite standard, according {\it e.g.} to the default
for printing programs: \\
{\tt  "1-3,5,7-9" } stands for {\tt "1,2,3,5,7,8,9"}, and so on.
The string {\tt "ALL"} is allowed as well, being equivalent to
{\tt "1-N$_{\text{max}}$"}. \\
{\sc default} = {\tt "ALL"} \par

}
%%%%%%%%%%%%%%%%%%%%%%%%%%%%%%%%%%%%%%%%%%%%%%%%%%%%%%%%%%%%%%%%%%%

\subsection{Input for post-processing calculations}
\subsubsection{bands.x}
\noindent Input file layout: \\

%
%
\begin{tabular}{l}
 {\tt \&INPUT} \\
   ... \\
   / \\
  \\
 {\tt label$\_1$ $\quad$ $k_1$ $k_2$ $k_3$ } \\
  ... \\
 {\tt label$\_N$ $\quad$ $k_1$ $k_2$ $k_3$ } \\
\end{tabular}
%
%
\\

\begin{centering}
\rule{2.2in}{0.01in} Namelist {\tt \&INPUT} \rule{2.2in}{0.01in}
\end{centering}\\

\newdimen\descindent \descindent = 8pc
{\noindent \leftskip = \descindent \parskip = .5\baselineskip
\llap{\hbox to \descindent{\tt prefix\hfil}}%
{\sc string} \\ the first part of the name of all the files written by the code;
              it should be equal to the value given in the main calculations.\\
{\sc default} = {\tt mandatory} \par

\noindent\llap{\hbox to \descindent{\tt postfix\hfil}}%
{\sc string} \\ the tail of the names of the above mentioned files (useful 
    {\it e.g.} to distinguish among different calculations having a common part);
    it should be equal to the value given in the main calculations.\\
    {\sc default} = {\tt ""} \par

\noindent\llap{\hbox to \descindent{\tt work\_dir\hfil}}%
{\sc string} \\ the scratch directory where the massive data files will be written.\\
              {\sc default} = {\tt "./"} \par

\noindent\llap{\hbox to \descindent{\tt verbosity\hfil}}%
{\sc string} \\ {\tt ( "low" $\mid$ "medium" $\mid$ "high" ) }\\
              the level of detail of the textual output file.\\
              {\sc default} =  {\tt "medium"} \par

\noindent\llap{\hbox to \descindent{\tt nkpts\_in\hfil}}%
{\sc integer} \\ number of edge $\mathbf{k}$-points defining the directions 
on which bands will be calculated.\\ 
{\sc default}: {\tt mandatory} \par

\noindent\llap{\hbox to \descindent{\tt nkpts\_max\hfil}}%
{\sc integer} \\ maximum number of interpolated $\mathbf{k}$-points. The actucal number
of points is calculated in the run and is written in the output file.\\ 
{\sc default} = {\tt 100} \par

\noindent\llap{\hbox to \descindent{\tt spin\_component\hfil}}%
{\sc string} \\  {\tt ( "up" $\mid$ "down" $\mid$ "none" ) }\\
              define whether the calculation is spin polarized and, if the case,
              which spin component is treated.\\ 
{\sc default} = {\tt "none"} \par
}\bigskip

\noindent After the {\tt \&INPUT} namelist, a description line for each of the 
{\tt nspts\_in} points must be provided. The format is as follows:\\

\noindent $\qquad$ \texttt{label} $\quad$ {\tt $k_1$ $k_2$ $k_3$ } \\

\newdimen\descindent \descindent = 8pc
{\noindent \leftskip = \descindent \parskip = .5\baselineskip
\llap{\hbox to \descindent{\tt label\hfil}}%
{\sc string} \\ a string with the conventional name of the $\mathbf{k}$-point. \par

\noindent\llap{\hbox to \descindent{\tt $k_1$ $k_2$ $k_3$\hfil}}%
{\sc real} \\ components of the $\mathbf{k}$-point vector in units of crystal reciprocal
              lattice \\
              ( {\it i.e.} $ \mathbf{k} = k_1 * \mathbf{b}_1 + k_2 * \mathbf{b}_2 + 
              k_3 * \mathbf{b}_3 $). \par
}\bigskip

\subsubsection{plot.x}
\noindent Input file layout: \\

%
%
\begin{tabular}{l}
  {\tt \&INPUT } \\
  ... \\
  / \\
\end{tabular}
%
%
\\

\begin{centering}
\rule{2.2in}{0.01in} Namelist {\tt \&INPUT} \rule{2.2in}{0.01in}
\end{centering}\\

\newdimen\descindent \descindent = 8pc
{\noindent \leftskip = \descindent \parskip = .5\baselineskip
\llap{\hbox to \descindent{\tt prefix\hfil}}%
{\sc string} \\ the first part of the name of all the file written by the code;
              it should be equal to the value given in the main calculations.\\
              {\sc default} = {\tt mandatory} \par

\noindent\llap{\hbox to \descindent{\tt postfix\hfil}}%
{\sc string} \\ the tail of the names of the above mentioned files. 
              It should be equal to the value given in the main calculations.\\
              {\sc default} = {\tt ""} \par

\noindent\llap{\hbox to \descindent{\tt work\_dir\hfil}}%
{\sc string} \\ the scratch directory where the massive data files will be written.\\
              {\sc default} = {\tt "./"} \par

\noindent\llap{\hbox to \descindent{\tt wann\hfil}}%
{\sc string} \\ specifies the indeces of the Wannier functions to be plotted.
              It is a string of format {\it e.g.} {\tt "1-3,5,7-9"} 
              (analogous to the fmt used to specify pages to standard print utilities)\\
              {\sc default} = {\tt mandatory} \par

\noindent\llap{\hbox to \descindent{\tt datatype\hfil}}%
{\sc string} \\ {\tt ("modulus" $\mid$ "real" $\mid$ "imaginary") }\\
              specifies the type of data plotted:\\
                {\tt "modulus"}:    plot the real space square modulus of the WFs. \\
                {\tt "real"}:       plot the real part (in real space) of the WFs. \\
                {\tt "imaginary"}:  plot the imaginary part (in real space) of the WFs
                              this choice shouldbe intended as a check because WFs
                              are expected to be more or less {\it real}.\\
              {\sc default} = {\tt "modulus"} \par

\noindent\llap{\hbox to \descindent{\tt output\_fmt\hfil}}%
{\sc string} \\ {\tt ( "plt" $\mid$ "txt" $\mid$ "cube" $\mid$ "xsf" ) }\\
              Define the format of the output files to be plotted. 
              {\tt "plt"} (read by {\sc gOpenMol}) is binary and smaller than 
              {\tt "cube"}, {\tt "xsf"} (read by {\sc xcrysden}) and
              {\tt "txt"}. While {\tt "cube"} and {\tt "xsf"} deal also 
              with non-orthorombic lattices, {\tt "txt"} is suitable to be converted 
              to further format.\\
              {\sc default} = {\tt "plt"} \par

\noindent\llap{\hbox to \descindent{\tt r1min, r1max\hfil}}%
{\sc real} \\ the starting and ending points of the plotting cell along the 
              first direct lattice vector $\mathbf{a}_1$.
              Units of $\mathbf{a}_1$.\\
              {\sc default} = {\tt -0.5, 0.5} \par

\noindent\llap{\hbox to \descindent{\tt r2min, r2max\hfil}}%
{\sc real} \\ as before but for $\mathbf{a}_2$ direction.\\
              {\sc default} = {\tt -0.5, 0.5} \par

\noindent\llap{\hbox to \descindent{\tt r3min, r3max\hfil}}%
{\sc real} \\ as before but for $\mathbf{a}_2$ direction.\\
              {\sc default} = {\tt -0.5, 0.5} \par

\noindent\llap{\hbox to \descindent{\tt assume\_ncpp\hfil}}%
{\sc logical} \\ if using norm-conserving pseudopot which are not readable 
               by \WANT\ set this value to .TRUE. in order to avoid their reading.\\
              {\sc default} = {\tt .FALSE.} \par

\noindent\llap{\hbox to \descindent{\tt locate\_wf\hfil}}%
{\sc logical} \\ if {\tt .TRUE.} moves the WFs in a unit cell centered around the 
    midpoint of the plotting cell. Useful to plot purposes.\\
              {\sc default} = {\tt .TRUE.} \par

\noindent\llap{\hbox to \descindent{\tt spin\_component\hfil}}%
{\sc string} \\ {\tt ( "up" $\mid$ "down" $\mid$ "none" ) }\\
              define whether the calculation is spin polarized and, if the case,
              which spin component is treated.\\
              {\sc default} = {\tt "none"} \par
}\bigskip

\subsubsection{blc2wan.x}
\noindent Input file layout: \\

%
%
\begin{tabular}{l}
  {\tt \&INPUT } \\
  ... \\
  / \\
\end{tabular}
%
%
\\

\begin{centering}
\rule{2.2in}{0.01in} Namelist {\tt \&INPUT} \rule{2.2in}{0.01in}
\end{centering}\\

\newdimen\descindent \descindent = 8pc
{\noindent \leftskip = \descindent \parskip = .5\baselineskip
\llap{\hbox to \descindent{\tt prefix\hfil}}%
{\sc string} \\ the first part of the name of all the file written by the code;
              it should be equal to the value given in the main calculations.\\
              {\sc default} = {\tt mandatory} \par

\noindent\llap{\hbox to \descindent{\tt postfix\hfil}}%
{\sc string} \\ the tail of the names of the above mentioned files. 
              It should be equal to the value given in the main calculations.\\
              {\sc default} = {\tt ""} \par

\noindent\llap{\hbox to \descindent{\tt work\_dir\hfil}}%
{\sc string} \\ the scratch directory where the massive data files will be written.\\
              {\sc default} = {\tt "./"} \par

\noindent\llap{\hbox to \descindent{\tt filein\hfil}}%
{\sc string} \\ the name of the file containing the input operator in the Bloch 
                representation.\\
              {\sc default} = {\tt mandatory} \par

\noindent\llap{\hbox to \descindent{\tt fileout\hfil}}%
{\sc string} \\ the name of the file containing the computed operator on the 
                Wannier basis.\\
              {\sc default} = {\tt mandatory} \par

\noindent\llap{\hbox to \descindent{\tt ascii\hfil}}%
{\sc logical} \\ if {\tt .TRUE.} the output file is written in textual XML format.\\
              {\sc default} = {\tt .FALSE.} \par
}
