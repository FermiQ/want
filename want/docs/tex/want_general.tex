%%%%%%%%%%%%%%%%%%%%%%%%%%%%%%%%%%%%%%%%%%%%%%%%%%%%%%%%%%
%  Copyright (C) 2005 WanT group                         %
%  This file is distributed under the terms of the       %
%  GNU General Public License.                           %
%  See the file `License'  in the root directory of      %
%  the present distribution,                             %
%  or http://www.gnu.org/copyleft/gpl.txt                %
%%%%%%%%%%%%%%%%%%%%%%%%%%%%%%%%%%%%%%%%%%%%%%%%%%%%%%%%%%

\thispagestyle{empty}
\begin{centering}
{\LARGE \WANT\ Version \WANTVERSION}\\
\end{centering}
\vspace{0.35in}

\noindent {\bf CREDITS.} The development and  maintenance of the
\WANT\ code is promoted by the National Research Center on
nanoStructures and bioSystems at Surfaces (S3) of the Italian
INFM-CNR (\texttt{http://www.s3.infm.it}) and the Physics Department North
Carolina State University (NCSU) (\texttt{http://ermes.physics.ncsu.edu}). \\

\noindent The present release of the \WANT\ package has been
realized by 
Andrea Ferretti (S3, MIT, Uni Oxford), Apr 2010. \\

\noindent The routines for the calculation of the
maximally-localized Wannier functions were originally written by
Nicola Marzari and David Vanderbilt (\copyright 1997);  Ivo Souza,
Nicola Marzari and David Vanderbilt (\copyright 2002). \\ 

\noindent The routines for the calculation of the quantum
conductance were originally written by Marco Buongiorno Nardelli
(\copyright 1998); Arrigo Calzolari, Nicola Marzari, and
Marco Buongiorno Nardelli (\copyright 2003).\\

\noindent For a full list of developers and contributors see the file
\texttt{\~{}want/docs/CREDITS }. \\
 \vspace{0.25in}

\noindent {\bf GENERAL DESCRIPTION.} \WANT\ is an open-source, GNU
General Public License suite of codes that provides an integrated
approach for the study of coherent electronic transport in
nanostructures. The core methodology combines state-of-the-art
Density Functional Theory (DFT), plane-waves, pseudopotential
calculations with a Green's functions method based on the Landauer
formalism to describe quantum conductance. The essential
connection between the two, and a crucial step in the calculation,
is the use of the maximally-localized Wannier function
representation to introduce naturally the ground-state electronic
structure into the lattice Green's function approach at the basis
of the evaluation of the quantum conductance. Moreover, the
knowledge of Wannier functions allows for a
direct link between the electronic transport properties of the
device with the nature of the chemical bonds, providing insight
into the mechanisms that govern electron flow at the nanoscale.\\

\noindent The \WANT\ package operates, in principles, as a simple
post-processing of any standard electronic structure code. The \WANT\ code
is currently interfaced to the codes: 
\begin{itemize}
 \item[\mydot]  \QUANTUMESPRESSO\ distribution (\QUANTUMESPRESSOURL).
 \item[\mydot]  \ABINIT\ (\ABINITURL).
 \item[\mydot]  \textsc{CRYSTAL09} ({\tt http://www.crystal.unito.it}).
 \item[\mydot]  \textsc{Sax} ({\tt http://www.sax-project.org}) .
\end{itemize}

\noindent \WANT\ capabilities include:
\begin{itemize}
\item[\mydot]   Calculation of maximally-localized Wannier functions (WF's);
\item[\mydot]   WF's interpolation of the electronic structure of a given system
                (calculation of bands, DOS, complex band structure).
\item[\mydot]   Calculation of polarization for periodic systems through the
                use of WFs (equivalent to the Berry phase formalism).
\item[\mydot]   Calculation of quantum conductance spectrum 
                for a lead-conductor-lead geometry. including
                eigenchannel analysis.
\item[\mydot]   Calculation of embedding dynamical and non-hermitean self-energies.
\end{itemize}

%\newpage
\vspace{0.25in}
\noindent{\bf TERMS OF USE.} Although users are not under any
obligation in the spirit of the GNU General Public Licence, the
developers of \WANT\ would appreciate the acknowledgment of the
effort to produce such codes in the form of the following
reference:\\

\noindent In the text: ``The results of this work have been
obtained using the \WANT\ package.[ref]''\\

\noindent In references: ``[ref] \WANT\ code by A. Ferretti, B. Bonferroni, A. Calzolari, 
and M. Buongiorno Nardelli, \texttt{http://www.wannier-transport.org}. 
\\
%See also: A. Calzolari, N. Marzari,
%. Souza and M. Buongiorno Nardelli, Phys. Rev. B 69, 035108 (2004).''\\
  \vspace{0.25in}

\noindent{\bf DISCLAIMER.} While the developers of \WANT\ make
every effort to deliver a high quality scientific software, we do
not guarantee that our codes are free from defects. Our software
is provided ``as is''. Users are solely responsible for determining
the appropriateness of using this package and assume all risks
associated with the use of it, including but not limited to the
risks of program errors, damage to or loss of data, programs or
equipment, and unavailability or interruption of operations. Due
to the limited human resources involved in the development of this
software package, no support will be given to individual users for
either installation or execution of the codes. Finally, in the
spirit of every open source project, any contribution from
external users is welcome, encouraged and, if appropriate, will be
included in future releases.\\
  \vspace{0.25in}

\noindent {\bf  LICENCE.} All the material included in this
distribution is free software; you can redistribute it and/or
modify it under the terms of the GNU General Public License as
published by the Free Software Foundation; either version 2 of the
License, or (at your option) any later version. These programs are
distributed in the hope that they will be useful, but WITHOUT ANY
WARRANTY; without even the implied warranty of MERCHANTABILITY or
FITNESS FOR A PARTICULAR PURPOSE. See the GNU General Public
License for more details. You should have received a copy of the
GNU General Public License along with this program; if not, write
to the Free Software Foundation, Inc., 675 Mass Ave, Cambridge, MA
02139, USA.
