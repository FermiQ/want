%%%%%%%%%%%%%%%%%%%%%%%%%%%%%%%%%%%%%%%%%%%%%%%%%%%%%%%%%%
%  Copyright (C) 2005 WanT group                         %
%  This file is distributed under the terms of the       %
%  GNU General Public License.                           %
%  See the file `License'  in the root directory of      %
%  the present distribution,                             %
%  or http://www.gnu.org/copyleft/gpl.txt                %
%%%%%%%%%%%%%%%%%%%%%%%%%%%%%%%%%%%%%%%%%%%%%%%%%%%%%%%%%%

\thispagestyle{empty}
\section{What to do when things go wrong?}
\label{section:troubleshot}
%
This section is not at all intended to be complete. Here we report a summary of the
most frequent and well-known problems in the day-by-day practice with \WANT,
and some tentative suggestions to solve them. 
Please, report any better solution or explaination you find to the maintainer of this
manual to make it more detailed.

%%%%%%%%%%%%%%%%%%%%%%%%%%%%%%%%%%%%%%%%%%%
\subsection{When do things go really wrong ?}
First it is necessary to understand which behaviors should be considered buggy and 
which may be conversely related to some failure of the implemented algorithms.
This section is devoted to guide the user to understand whether the code is
properly running or not. \\
%
%
\begin{itemize}
\item   {\bf Stops and crashes.} \\
        When the code stops, it is expected or to have reached the end
        of the calculation (which is signed in the output file by a summary
        of the timing) or to print
        out an error message and give a fortran stop. Any different behavior
        should be considered a bug and should be reported 
        (obviously it may be related to machine dependent problems, 
        indipendently of \WANT).

\item   {\bf Problems connected to memory usage.} \\
        Since the code is still serial, no distribution of the memory among the
        CPU is performed and crashes may result. In these cases, both an
        error message during allocations or a more drastic error (even segmentation
        fault) may occur. The largest amount of memory is used by 
        {\tt disentangle.x} (which manage wave-functions), while {\tt wannier.x} 
        does not require so much. Post-processing ({\tt bands.x, blc2wan.x}) 
        do not have special requirement too, while {\tt plot.x} needs a sensible
        amount of memory, about half of {\tt disentangle.x}.
       
\item   {\bf Convergence in disentanglement minimization.} \\
        The {\bf iterative procedure} in {\tt disentangle.x} is quite robust. 
        Non-monotonic behaviors of the invariant spread are unexpected
        and are probably related to algorith failures. Although, convergence
        maybe very slow, especially when some parameters are not properly set
        (too many empty states in the main energy window, strange numbers of 
        requested WFs...) .

\item   {\bf Convergence in Wannier localization.} \\
        In the case of {\tt wannier.x}, convergence is less strightfarward.
        Particularly for large systems, the behavior of the total spread 
        is typically decreasing for a number of iterations, then it suddenly 
        junps to a higher value and after it keeps decreasing. This cycle may be 
        repeated several times.
        From the expressions~\cite{nicola} of the spread functional, we see that
        many evaluation of the immaginary part of complex logarithms ({\it i.e.} 
        the pases of the arguments, which iin the present case are the overlap 
        integrals) are needed. Since logarith in the complex plane
        is a multivalued function, sudden junps in the total spread (towards larger 
        values) can be related to changes in the logarith branch ({\it e.g.} a phase 
        which is increasing from $0$ to $2\pi$ is suddenly taken to 0 (
        {\it i.e.} to a different branch) when corssing the $2\pi$ value.
        The problem becomes more evident when a large number of $\mathbf{k}$-points
        is used for the sampling of the Brillouin zone.

        Such behaviors should be considered ordinary: the user is suggested to increase
        the maximum number of iterations (50000 steps may be fully acceptable) 
        when running {\tt wannier.x} . Since this problem is mainly connected with 
        large $\mathbf{k}$-point mesh, it should luckily leave unaffected large scale
        calculations (where we usually have large number of bands in a very small BZ).

\item   {\bf How can I understand when my Wannier functions are well behaved ?} \\
        Directly from the definition of {\it maximally localized} WFs, we basically
        are interested in {\it localized} orbitals spanning the original subspace
        of Bloch states. Our measure of localization (the spread functional $\Omega$) 
        is anyway a global property of the WF set as a whole. 
        This means that even if we are reaching lower and lower
        values of the spread, the set of WFs we found is a good one if {\it each} 
        single WF is localized.
        
        If our application of WFs is based on their localization property, then 
        a single WF not properly localized can make the all set useless. This is
        exactly what happens in the current application to transport (but it is not 
        for instance the case of the calculation of spontaneus polaritation, which is 
        a property of the occupied manyfold).
        
        Once we attach the idea of {\it good behavior} to that of localization of each WF,
        it is possible to identify the following criteria XXX: 
        %
        %
        \begin{itemize}
        \item    the spread of each WF should be in a reasonable range 
                 ( < 15--20 Bohr$^2$). Note that the current version of \WANT fully 
                 adopts Bohr units, while older versions ({\it e.g. v1.0}) use $\AA^2$ 
                 for the spread in the {\tt wannier.x} output file.
        \item    during a typical run, the obtained WFs have different characters 
                 (spreads, shapes... ). 
        \end{itemize}
        
                 
        
       


\end{itemize}


%%%%%%%%%%%%%%%%%%%%%%%%%%%%%%%%%%%%%%%%%%%
\subsection{Troubleshotting (sort of)}


