%%%%%%%%%%%%%%%%%%%%%%%%%%%%%%%%%%%%%%%%%%%%%%%%%%%%%%%%%%
%  Copyright (C) 2005 WanT group                         %
%  This file is distributed under the terms of the       %
%  GNU General Public License.                           %
%  See the file `License'  in the root directory of      %
%  the present distribution,                             %
%  or http://www.gnu.org/copyleft/gpl.txt                %
%%%%%%%%%%%%%%%%%%%%%%%%%%%%%%%%%%%%%%%%%%%%%%%%%%%%%%%%%%

\thispagestyle{empty}
\section{What to do whan things go wrong?}
\label{section:troubleshot}
%
This section is not at all intended to be complete. Here we report a summary of the
most frequent and well-known problems in the day-by-day practice with \WANT,
and some tentative suggestions to solve them. 
Please, report any better solution or explaination you find to the maintainer of this
manual to make it more detailed.

%%%%%%%%%%%%%%%%%%%%%%%%%%%%%%%%%%%%%%%%%%%
\subsection{When do things go really wrong ?}
First it is necessary to understand which behaviors should be considered buggy and 
which may be conversely related to some failure of the implemented algorithms.
This section is devoted to guide the user to understand whether the code is
properly running or not. \\
%
%
\begin{itemize}
\item   When the code stops, it is expected or to have reached the end
        of the calculation (which is signed in the output file by a summary
        of the timing) or to print
        out an error message and give a fortran stop. Any different behavior
        should be considered a bug and should be reported 
        (obviously it may be related to machine dependent problems, 
        indipendently of \WANT).

\item   Since the code is still serial, no division of the memory among the
        CPU is performed and crashes may result. In these cases, both an
        error message during allocations or a more drastic error (even segmentation
        fault) may occur. The largest amount of memory is used by 
        {\tt disentangle.x} (which manage wave-functions), while {\tt wannier.x} 
        does not require so much. Post-processing ({\tt bands.x, blc2wan.x}) 
        do not have special requirement too, while {\tt plot.x} needs a sensible
        amount of memory, about half of {\tt disentangle.x}.
       
\item   The iterative procedure in {\tt disentangle.x} is quite robust. 
        Non-monotonic behaviors of the invariant spread are unexpected
        and are probably related to algorith failures. Although, convergence
        maybe very slow, especially when some parameters are not properly set
        (too many empty states in the main energy window, strange numbers of 
        requested WFs...) .

\item   In the case of {\tt wannier.x}, convergence is not strightfarward.

\end{itemize}


%%%%%%%%%%%%%%%%%%%%%%%%%%%%%%%%%%%%%%%%%%%
\subsection{Troubleshotting (sort of)}


